%-------------------------------------------------------------------------------
% This file contains the title, author and abstract.
% It also contains all relevant document numbers used for an ATLAS note.
%-------------------------------------------------------------------------------

% Title
\AtlasTitle{A Deep Learning Approach to Differential Measurements of Higgs - Top Interactions in Multilepton Final States using the ATLAS Detector at the LHC}

% Draft version:
% Should be 1.0 for the first circulation, and 2.0 for the second circulation.
% If given, adds draft version on front page, a 'DRAFT' box on top of each other page, 
% and line numbers.
% Comment or remove in final version.
\AtlasVersion{0.1}

% Abstract - % directly after { is important for correct indentation
\AtlasAbstract{%
        Several theories Beyond the Standard Model predict a modification of the momentum spectrum of the Higgs Boson, without a significantly altered rate of Higgs produced in association with top quark pairs ($t\bar{t}H$). This provides a physical observable that can be used to search for new physics based on data collected by the LHC. This thesis presents techniques and preliminary results for a differential measurement of the Higgs transverse momentum in $t\bar{t}H$ events with multiple leptons in the final state, using data collected at an energy of $\sqrt{s}$ = 13 TeV by the ATLAS detector at the LHC.

This thesis also details a measurement of WZ + heavy flavor production, a significant background to $t\bar{t}H$ that is poorly understood. This study targets events with three leptons and one or two jets in the final state, using 140 $fb^{-1}$ of  $\sqrt{s}$ = 13 TeV data. A measured cross-section of $X\pm X$ fb ($X\pm X$ fb) is observed for WZ + b (WZ + charm) with 1 associated jet and $X\pm X$ fb ($X\pm X$ fb) for WZ + b (WZ + charm) with 2 assoicated jets.

Because of the challenges inherent in reconstructing the Higgs in multilepton final states, a deep learning approach is used to predict the momentum spectrum of the Higgs for these events. The regressed Higgs $p_T$ spectrum is fit to data for events with two same-sign leptons or three leptons in the final state. The fit is used to extract normalization factors for high ($p_{T}(H) > 150$ GeV) and low ($p_{T}(H) < 150$ GeV) momentum $t\bar{t}H$ events. Preliminary results are presented for 80 $fb^{-1}$ of data, with projected results shown for 140 $fb^{-1}$.

 }

% Author - this does not work with revtex (add it after \begin{document})
\author{The ATLAS Collaboration}

% Authors and list of contributors to the analysis
% \AtlasAuthorContributor also adds the name to the author list
% Include package latex/atlascontribute to use this
% Use authblk package if there are multiple authors, which is included by latex/atlascontribute
% \usepackage{authblk}
% Use the following 3 lines to have all institutes on one line
% \makeatletter
% \renewcommand\AB@affilsepx{, \protect\Affilfont}
% \makeatother
% \renewcommand\Authands{, } % avoid ``. and'' for last author
% \renewcommand\Affilfont{\itshape\small} % affiliation formatting
% \AtlasAuthorContributor{First AtlasAuthorContributor}{a}{Author's contribution.}
% \AtlasAuthorContributor{Second AtlasAuthorContributor}{b}{Author's contribution.}
% \AtlasAuthorContributor{Third AtlasAuthorContributor}{a}{Author's contribution.}
% \AtlasContributor{Fourth AtlasContributor}{Contribution to the analysis.}
% \author[a]{First Author}
% \author[a]{Second Author}
% \author[b]{Third Author}
% \affil[a]{One Institution}
% \affil[b]{Another Institution}

% If a special author list should be indicated via a link use the following code:
% Include the two lines below if you do not use atlasstyle:
% \usepackage[marginal,hang]{footmisc}
% \setlength{\footnotemargin}{0.5em}
% Use the following lines in all cases:
% \usepackage{authblk}
% \author{The ATLAS Collaboration%
% \thanks{The full author list can be found at:\newline
%   \url{https://atlas.web.cern.ch/Atlas/PUBNOTES/ATL-PHYS-PUB-2017-007/authorlist.pdf}}
% }

% ATLAS reference code, to help ATLAS members to locate the paper
\AtlasRefCode{GROUP-2017-XX}

% ATLAS note number. Can be an COM, INT, PUB or CONF note
% \AtlasNote{ATLAS-CONF-2017-XXX}
% \AtlasNote{ATL-PHYS-PUB-2017-XXX}
% \AtlasNote{ATL-COM-PHYS-2017-XXX}
