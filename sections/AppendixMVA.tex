
The following section provides details regarding various studies performed in support of this analysis, exploring alternate decisions and strategies. 

%---------------------------------------------------------------------- 
\subsection{Alternate b-jet Identification Algorithm}
\label{subsec:topRecoApx}
%---------------------------------------------------------------------- 

The nominal analysis reconstructs the b-jets by considering different combinations of jets, and asking a neural network to determine whether each combination consists of b-jets from top quark decays. An alternate approach would be to give the neural network about all of the jets in an event at once, and train it to select which two are most likely to be the b-jets from top decay. It was hypothesized that this could perform better than considering each combination independently, as the neural network could consider the event as a whole. While this is not found to be the case, these studies are documented here as a point of interest and comparison.

For these studies, the kinematics of the 10 highest $p_T$ jets in each event are used for training. This includes the vast majority of truth b-jets. Specifically the $p_T$, $\eta$, $\phi$, $E$, and DL1r score of each jet are used. For events with fewer than 10 jets, these values are substituted with 0. The $p_T$, $\eta$, $\phi$, and $E$ of the leptons and $E_T^{miss}$ are included as well. 

Caterogical cross entropy is used as the loss function.

\begin{table}[h!]
  \centering
  \caption{Accuracey of the NN in identifying b-jets from tops in 2lSS events for the alternate categorical method compared to the nominal method.}
  \begin{tabular}{l|c|c}                                                                                                     
  \hline\hline
  Channel & Categorical & Nominal \\                                                                              
  \hline                                                                                                                  
  2lSS & & 73.9\% \\
  3l & & 79.8\% \\
  \hline                                                                                                                 
  \end{tabular}                                                                                                           
  \label{tab:topMatchCatApx}
\end{table}

%---------------------------------------------------------------------- 
\subsubsection{Binary Classification of the Higgs $p_T$}
\label{subsec:binPtApx}
%---------------------------------------------------------------------- 
                                                                                                                     
A two bin fit of the Higgs momentum is used because statistics are insufficient for any finer resolution. This means separating high and low $p_T$ events is sufficient for this analysis. As such, rather than attempting the reconstruct the full Higgs $p_T$ spectrum, a binary classification approach is explored.

A model is built to determine whether $t\bar{t}H$ events include a high $p_T$ ($>$150 GeV) or low $p_T$ ($<$150 GeV) Higgs Boson. While this is now a classification model, it uses the same input features described in section \ref{sec:ptReco}. Binary crossentropy is used as the loss function.

\begin{figure}[h!]                                                                                                           
  \subfigure[]{\includegraphics[width=.48\linewidth]{ptReco/higgsTop2lSS/bin_keras_score.png}}%                
  \subfigure[]{\includegraphics[width=.48\linewidth]{ptReco/higgsTop2lSS/bin_keras_roc.png}}\\
  \caption{}
  \label{fig:bin2lSSroc}                                                                          
\end{figure}

\begin{figure}[h!]
  \subfigure[]{\includegraphics[width=.48\linewidth]{ptReco/higgsTop3lS/bin_keras_score.png}}%                               
  \subfigure[]{\includegraphics[width=.48\linewidth]{ptReco/higgsTop3lS/bin_keras_roc.png}}\\                                
  \caption{}
  \label{fig:bin3lSroc}
\end{figure} 

\begin{figure}[h!]
  \subfigure[]{\includegraphics[width=.48\linewidth]{ptReco/higgsTop3lF/bin_keras_score.png}}%                               
  \subfigure[]{\includegraphics[width=.48\linewidth]{ptReco/higgsTop3lF/bin_keras_roc.png}}\\                                
  \caption{}
  \label{fig:bin3lFroc}
\end{figure} 

%----------------------------------------------------------------------
\subsubsection{Impact of Alternative Jet Selection}
\label{subsec:ptCutApx}
%---------------------------------------------------------------------- 

A relatively low $p_T$ threshold of 15 GeV is used to determine jet candidates, as the jets originating from the Higgs decay are found to fall between 15 and 25 GeV a large fraction of the time. The impact of different jet $p_T$ cuts on our ability to reconstruct the Higgs $p_T$ is explored here. The performance of the Higgs $p_T$ prediction models is evaluated for jet $p_T$ cuts of 10, 15, 20, and 25 GeV.

\begin{figure}[h!]
    \subfigure[]{\includegraphics[width=.48\linewidth]{ptReco/higgsTop2lSS_20/keras_test_pt_scatter.png}}%            
    \subfigure[]{\includegraphics[width=.48\linewidth]{ptReco/higgsTop2lSS_20/keras_test_norm2Dhist.png}}\\           
    \subfigure[]{\includegraphics[width=.48\linewidth]{ptReco/higgsTop2lSS_20/keras_score.png}}%   
    \subfigure[]{\includegraphics[width=.48\linewidth]{ptReco/higgsTop2lSS_20/keras_roc.png}}\\                       
    \caption{}
    \label{fig:pt2lSS_20}
\end{figure} 

%----------------------------------------------------------------------
