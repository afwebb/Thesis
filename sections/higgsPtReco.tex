Reconstructing the momentum of the Higgs boson is a particular challenge for channels with leptons in the final state: Because all channels include at least two neutrinos in the final state, the Higgs can never be fully reconstructed. However, the momentum spectrum can be well predicted by a neural network when provided with the four-vectors of the Higgs Boson decay products, as shown in section \ref{sec:truthLevelReco}. With this in mind, several layers of MVAs are used to reconstruction the Higgs momentum. 

The first layer is a model designed to select which jets are most likely to be the b-jets that came from the top decay, detailed in section \ref{sec:bjetID}. As described in section \ref{sec:higgsID},the kinematics of these jets are fed into the second layer, which is designed to identify the decay products of the Higgs Boson itself. The kinematics of these particles are then fed into yet another neural-network, which predicts the momentum of the Higgs (\ref{sec:ptReco}). MVAs are also used in the analysis to determine the decay of the Higgs boson in the 3l channel (\ref{sec:decay3l}). 

For all of these models, the Keras neural network framework, with Tensorflow as the backend, is used, and the number of hidden layers and nodes are determined using grid search optimization. Each neural network uses the LeakyReLU activation function, a learning rate of 0.01, and the Adam optimization algorithm, as alternatives are found to either decrease or have no impact on performance. For the classification algorithms (b-jet matching, Higgs reconstruction, and 3l decay identification) binary-cross entropy is used as the loss function, while the $p_T$ reconstruction algorithm uses MSE. 

The specific inputs features used for each model are arrived at through a process of trial and error - features considered potentially useful are tried, and those that are found to increase performance are included. While each model includes a relatively large number of features, some using upwards of 30, this inclusive approach is found to maximize the performance of each model while decreasing the variance compared to a reduced number of inputs. Each input feature is validated by comparing MC simulations to 80 $fb^{-1}$ of data, as shown in the sections below.

\subsection{Decay Candidate Reconstruction}
\label{sec:truthLevelReco}

Machine Learning algorithms are trained to identify the decay products of the Higgs Boson using MC simulations of $t\bar{t}H$ events. These include light leptons and jets. Reconstructed physics objects are matched to truth level particles, in order to identify the parents of these reconstructed objects. The kinematics of the decay product candidates as well as event level variables are used as inputs. 

Leptons considered as possible Higgs and top decay candidates are required to pass the selection described in section \ref{subsec:lepSelection}. For jets, however, it is found that a large fraction that originate from either the top decay or the Higgs decay fall outside the selection described in section \ref{subsec:jetSelection}. Specifically, jets from the Higgs decay tend to be soft, with 37\% having $p_T$ $<$ 25 GeV. Therefore jets with $p_T$ $<$ 10 GeV are considered as possible candidates in the models described below. The spectrum is found to be well modeled even down to this low $p_T$ threshold, as shown in section \ref{subsec:preMVA}. As they are expected to originate from the primary vertex, jets are also required to pass a JVT cut.

%------------------------------------------------------------------------ 
\subsection{b-jet Identification}
\label{sec:bjetID}
%------------------------------------------------------------------------ 

Including the kinematics of the b-jets that originate from the top decay is found to improve the identification of the Higgs decay products, and improve the accuracey with which the Higgs momentum can be reconstructed. Because these b-jets are reconstructed by the detector with high efficiency (just over 90\% of the time), and can be identified relatively consistently, the first step in reconstructing the Higgs is selecting the b-jets from the top decay.

Exactly two b-jets are expected in the final state of $t\bar{t}H-ML$ events. However, in both the 3l and 2lSS channels, only one or more b-tagged jets are required (where the 70\% DL1r b-tag working point is used). Therefore, for events which have exactly one, or more than two, b-tagged jets, deciding which combination of jets correspond to the top decay is non-trivial. Further, events with 1 b-tagged jet represent just over half of all $t\bar{t}H-ML$ events. Of those, both b-jets are reconstructed by the detector ~70\% of the time. Therefore, rather than adjusting the selection to require exactly 2 b-tagged jets, and losing more than half of the signal events, a neural network is used to predict which pair of jets is most likely to correspond to truth b-jets.

Once the network is trained, all possible pairings of jets are fed into the model, and the pair of jets with the highest output score are taken to be b-jets in successive steps of the analysis. 

%----------------------------------------------------------------------                                                  
\subsubsection{2lSS Channel}
\label{subsec:top2lSS}
%----------------------------------------------------------------------                                                  

For the 2lSS channel, the input features shown in table \ref{tab:top2lSSfeatures} are used for training. 

\begin{table}[h!]
  \begin{center}
  \begin{tabular}{ccc}
    $M(j_0j_1)$ & $M(j_0j_1l_0l_1E_T^{miss})$ & $M(l_0j_0)$ \\
    $M(l_0j_1)$ & $M(l_1j_0)$ & $M(l_1j_1)$ \\
    $p_T(j_0j_1l_0l_1E_T^{miss})$ & $\Delta\phi(j_0)(E_T^{miss})$ & $\Delta\phi(j_1)(E_T^{miss})$ \\
    $\Delta R(j_0)(j_1)$ & $\Delta R(j_0l_0)(j_1l_1)$ & $\Delta R(j_0l_1)(j_1l_0)$ \\
    $\Delta R(l_0)(j_0)$ & $\Delta R(l_0)(j_1)$ & $\Delta R(l_1)(j_0)$ \\
    $\Delta R(l_1)(j_1)$ & jet DL1r WP 0 & jet DL1r WP 1 \\
    jet  $\eta$ 0 & jet  $\eta$ 1 & jet  $p_T$ 0 \\
    jet  $p_T$ 1 & Lepton  $p_T$ 0 & Lepton  $p_T$ 1 \\
    nJets & nJets OR DL1r 60 & nJets OR DL1r 85\\
    $E_T^{miss}$ & &\\
  \end{tabular}
  \end{center}
  \caption{Input features used in the 2lSS b-jet identification algorithm}
  \label{tab:top2lSSfeatures}
\end{table}

The difference between the distributions for a few of these features for the \"correct\" (i.e. both jets are truth b-jets), and \"incorrect\" combinations are shown in figure \ref{fig:features_top2lSS}.

\begin{figure}
    \subfigure[]{\includegraphics[width=.29\linewidth]{topMatching/top2lSS/features/jet_DL1r_0.pdf}}%                 
    \subfigure[]{\includegraphics[width=.29\linewidth]{topMatching/top2lSS/features/jet_Pt_0.pdf}}%                    
    \subfigure[]{\includegraphics[width=.29\linewidth]{topMatching/top2lSS/features/dR_j0_j1.pdf}}\\
    \subfigure[]{\includegraphics[width=.29\linewidth]{topMatching/top2lSS/features/Ml1j1.pdf}}%                    
    \subfigure[]{\includegraphics[width=.29\linewidth]{topMatching/top2lSS/features/dR_l0_j0.pdf}}%                    
    \subfigure[]{\includegraphics[width=.29\linewidth]{topMatching/top2lSS/features/Ptj0j1l0l1met.pdf}}\\
    \caption{Input features for top2lSS training}
    \label{fig:features_top2lSS}                                                                                        
\end{figure}

The modeling of these inputs is validated against data, with figure \ref{fig:model_top2lSS} showing good general agreement between data and MC. Plots for the complete list of features can found in section \ref{apx:MVA}.

\begin{figure}
    \subfigure[]{\includegraphics[width=.29\linewidth]{trexPlots/top2lSSfeatures/Plots/jet_DL1r_0.png}}%                  
    \subfigure[]{\includegraphics[width=.29\linewidth]{trexPlots/top2lSSfeatures/Plots/jet_Pt_0.png}}%                
    \subfigure[]{\includegraphics[width=.29\linewidth]{trexPlots/top2lSSfeatures/Plots/dR_j0_j1.png}}\\
    \subfigure[]{\includegraphics[width=.29\linewidth]{trexPlots/top2lSSfeatures/Plots/Ml1j1.png}}%                      
    \subfigure[]{\includegraphics[width=.29\linewidth]{trexPlots/top2lSSfeatures/Plots/dR_l0_j0.png}}%                   
    \subfigure[]{\includegraphics[width=.29\linewidth]{trexPlots/top2lSSfeatures/Plots/Ptj0j1l0l1met.png}}\\
    \caption{Modeling of input features for top2lSS training}
    \label{fig:model_top2lSS}
\end{figure}

Based on the results of grid search evaluation, the optimal architecture is found to include 5 hidden layers with 40 nodes each. The output score distribution as well as the ROC curve for the trained model are shown in figure \ref{fig:top2lSSresults}. The model is found to identify the correct pairing of jets for 73\% of 2lSS signal events on test data.

\begin{figure}
  \subfigure[]{\includegraphics[width=0.48\linewidth]{topMatching/top2lSS/keras_score.png}}%  
  \subfigure[]{\includegraphics[width=0.48\linewidth]{topMatching/top2lSS/keras_roc.png}}
  \caption{(left) the output score of the NN for correct and incorrect combinations of jets. (right) the ROC curve...}
\end{figure}

For point of comparison, a \"naive\" approach to identify b-jets is used as well: The two jets which pass the highest DL1r b-tag working point are assumed to be the b-jets from the top decay. In the case that multiple jets meet the same b-tag working point, the jet with higher $p_T$ is used. This method identifies the correct jet pair 65\% of the time. 

The accuracey of the model for different values of n-bjets, compared to this naive approach, is shown in table \ref{tab:topMatchAcc2lSS}.

\begin{table}[h]
  \centering
  \caption{Accuracey of the NN in identifying b-jets from tops in 2lSS events.}
  \hline
  2lSS Channel
  \begin{tabular}{l|c|c}
    \hline\hline
    & NN & Naive \\
    \hline
    1 b-jet & & \\
    2 b-jets & & \\
    $>=$3 b-jets & & \\
    \hline
    Overall & 78.8\% & 65.2\% \\                                                                                  
    \hline\hline                                                                                                         
  \end{tabular}
  \label{tab:topMatchAcc2lSS}                                                                                           
\end{table}

%----------------------------------------------------------------------                                                  
\subsubsection{3l Channel}
\label{subsec:top3l}
%----------------------------------------------------------------------                                                     
 
Based on the results of grid search evaluation, the optimal architecture is found to include 5 hidden layers with 40 nodes each. The output score distribution as well as the ROC curve for the trained model are shown in figure \ref{fig:top3lresults}.

\begin{figure}                                                                                                           
   \subfigure[]{\includegraphics[width=0.48\linewidth]{topMatching/top3l/keras_score.png}}%   
   \subfigure[]{\includegraphics[width=0.48\linewidth]{topMatching/top3l/keras_roc.png}}                             
   \label{fig:top3lresults}                                                                                               
   \caption{tmp}
\end{figure}

This procedure is found to identify the correct pairing of jets for 73\% of 2lSS signal events, and 79\% of 3l signal events, for test sets of data separate from the dataset used to train the model. 

For point of comparison, a \"naive\" approach to identify b-jets is used as well: The two jets which pass the highest DL1r b-tag working point are assumed to be the b-jets from the top decay. In the case that multiple jets meet the same b-tag working point, the jet with higher $p_T$ is used. This method identifies the correct jet pair 65\% of the time for the 2lSS channel and 68\% of the time in the 3l channel.  

The accuracy of the 

\begin{table}[h]
\centering
\caption{Accuracey of the NN in identifying b-jets from tops.}

\begin{tabular}{l|c|c}
\hline
%2lSS Channel
\hline\hline
& NN & Naive \\
\hline
1 b-jet & & \\
2 b-jets & & \\
$>=$3 b-jets & & \\
\hline
Overall & 78.8\% & 70.2\% \\
\hline\hline
\end{tabular}

\begin{tabular}{l|c|c}
\hline
%3l Channel
\hline\hline
& NN & Naive \\
\hline
1 b-jet & & \\
1 b-jet & & \\
2 b-jets & & \\                                                                                                            
$>=$3 b-jets & & \\ 
\hline
Overall & 78.8\% & 70.2\% \\
\hline\hline
\end{tabular}

\label{tab:topMatchAcc}
\end{table}




%------------------------------------------------------------------------ 

%------------------------------------------------------------------------
\subsection{Higgs Reconstruction}
\label{sec:higgsID}
%------------------------------------------------------------------------ 

Techniques similar to the b-jet identification algorithms are employed to select the decay products of the Higgs. 

%------------------------------------------------------------------------ 

%------------------------------------------------------------------------ 
\subsection{$p_T$ Prediction}
\label{sec:ptReco}
%------------------------------------------------------------------------ 

Once the most probable decay products have been identified, their kinematics are used to reconstruct the momentum spectrum of the Higgs Boson. 

%------------------------------------------------------------------------ 
\subsection{3l Decay Mode}
\label{sec:decay3l}
%------------------------------------------------------------------------ 

In the 3l channel, there are two possible ways for the Higgs to decay, both involving intermediate W boson pairs: Either both W bosons decay leptonically, in which case the reconstructed decay consists of two leptons (referred as the fully-leptonic 3l channel), or one W decays leptonically and the other hadronically, giving two jets and one lepton in the final state (referred to as the semi-leptonic 3l channel). In order to accurately reconstruct the Higgs, it is necessary to identify which of these decays took place for each 3l event.

%------------------------------------------------------------------------ 


