Events are divided into two channels based on the number of leptons in the final state: one with two same-sign leptons, the other with three leptons. The $3l$ channel includes events where both leptons originated from the Higgs boson as well as events where only one of the leptons 

%------------------------------------------------------------------------------------------

\subsection{Pre-MVA Event Selection}
\label{subsec:preMVA}

A preselection is applied to define orthogonal analysis channels based on the number of leptons in each event. For the 2lSS channel, the following presection is used:

\begin{itemize}
  \item Two very tight, same-charge, light leptons with $p_T > 20$ GeV
  \item $\geq$4 reconstructed jets, $\geq$1 b-tagged jets
  \item No reconstructed tau candidates
\end{itemize}

The event yield after the 2lSS preselection has been applied, for MC and data at 79.8 $fb^{-1}$, is shown in Table \ref{tab:Yields2lSS}. 

\begin{center}
\begin{tabular}{|l|c|}
\hline 
Process & {Yield}\\
\hline 
  $t\bar{t}H$\ high\ $p_T$   & 41 $\pm$ 5 \\ 
  $t\bar{t}H$\ low\ $p_T$   & 71 $\pm$ 8 \\ 
  $t\bar{t}W$   & 450 $\pm$ 70 \\ 
  $t\bar{t}(Z/\gamma^*)$   & 91 $\pm$ 11 \\ 
  $t\bar{t}ll$ low mass   & 10 $\pm$ 6 \\ 
  Rare Top   & 20 $\pm$ 12 \\ 
  $VV$   & 42 $\pm$ 22 \\ 
  $tZ$   & 10 $\pm$ 5 \\ 
  QMisID   & 44.7 $\pm$ 2.7 \\ 
  Fakes int. conv & 47 $\pm$ 26 \\
  Fakes ext. conv & 46 $\pm$ 44 \\
  Fakes HF e & 45 $\pm$ 23 \\
  Fakes HF $\mu$ & 250 $\pm$ 50 \\
  Three top   & 2.2 $\pm$ 1.1 \\ 
  Four top   & 5.64 $\pm$ 0.31 \\ 
  $t\bar{t}WW$   & 10.9 $\pm$ 0.6 \\ 
  $tW$   & 0.0 $\pm$ 0.0 \\ 
  $WtZ$   & 9.1 $\pm$ 0.8 \\ 
  $VVV$   & 0.30 $\pm$ 0.05 \\ 
  $VH$   & 0.6 $\pm$ 1.0 \\ 
\hline 
  Total  & 1170 $\pm$ 120 \\ 
\hline 
  Data   & 1108 \\ 
\hline 
\end{tabular} 
\end{center} 


Figure \ref{fig:presel2lSS}. Good general agreement is found.

\begin{figure}[H]
    \centering
    \subfigure[]{\includegraphics[width=.29\linewidth]{trexPlots/sys2l_80/Plots/lep_Pt_0.png}}%                        
    \subfigure[]{\includegraphics[width=.29\linewidth]{trexPlots/sys2l_80/Plots/lep_Pt_1.png}}%                     
    \subfigure[]{\includegraphics[width=.29\linewidth]{trexPlots/sys2l_80/Plots/Mll01}}\\
    \subfigure[]{\includegraphics[width=.29\linewidth]{trexPlots/sys2l_80/Plots/nJets.png}}%                       
    \subfigure[]{\includegraphics[width=.29\linewidth]{trexPlots/sys2l_80/Plots/nbJets.png}}%                
    \subfigure[]{\includegraphics[width=.29\linewidth]{trexPlots/sys2l_80/Plots/MET.png}}\\
    \caption{Data/MC comparisons of the $2lSS$ pre-selection region. (a) and (b) show the $p_T$ of leptons 0 and 1, (c) shows the invariant mass of lepton 0 and 1, (d) shows the jet multiplicity, (e) the b-tagged jet multiplicity, and (f) the missing transverse energy.}                           
    \label{fig:presel2lSS}
\end{figure}

For the 3l channel, the following selection is applied:

\begin{itemize}
  \item Three light leptons with total charge $\pm 1$
  \item Same charge leptons are required to be very tight, with $p_T > 20$ GeV
  \item Opposite charge lepton must be loose, with $p_T > 10$ GeV
  \item $\geq$2 reconstructed jets, $\geq$1 b-tagged jets                                              
  \item No reconstructed tau candidates
  \item $|M(l^+l^-)-91.2\textrm{ GeV}| > 10$~\GeV{} for all opposite-charge, same-flavor lepton pairs
\end{itemize}

The event yield after the 3l preselection has been applied, for MC and data at 79.8 $fb^{-1}$, is shown in Table \ref{tab:Yields3l}.

\begin{center}
\begin{tabular}{|l|c|}
\hline 
Process & {Yield}\\
\hline 
  $t\bar{t}H$ high $p_T$   & 20.5 $\pm$ 2.3 \\ 
  $t\bar{t}H$ low $p_T$   & 33.6 $\pm$ 3.8 \\ 
  $t\bar{t}W$   & 138 $\pm$ 18 \\ 
  $t\bar{t}Z/\gamma$   & 80 $\pm$ 9 \\ 
  $t\bar{t}ll low mass$   & 3.5 $\pm$ 2.0 \\ 
  rareTop   & 22 $\pm$ 12 \\ 
  $VV$   & 39 $\pm$ 19 \\ 
  $tZ$   & 9.2 $\pm$ 4.5 \\ 
  QMisID   & 1.8 $\pm$ 0.6 \\ 
  Fakes int. conv & 31 $\pm$ 17 \\
  Fakes ext. conv & 14 $\pm$ 11 \\
  Fakes HF e & 20 $\pm$  10 \\
  Fakes HF $\mu$ & 102 $\pm$ 22 \\  
  Three top   & 0.96 $\pm$ 0.48 \\ 
  Four top   & 6.17 $\pm$ 0.35 \\ 
  $t\bar{t}WW$   & 5.46 $\pm$ 0.33 \\ 
  $tW$   & 0.0 $\pm$ 0.0 \\ 
  $WtZ$   & 8.7 $\pm$ 0.6 \\ 
  $VVV$   & 0.81 $\pm$ 0.11 \\ 
  $VH$   & 0.0 $\pm$ 0.0 \\ 
\hline 
  Total  & 512 $\pm$ 48 \\ 
\hline 
  Data   & 535 \\ 
\hline 
\end{tabular} 
\label{tab:Yields3l}
\caption{Yields of the $3l$ preselection region.} 
\end{center} 



Comparisons of kinematic distributions for data and MC in this region are shown in Figure \ref{fig:presel3l}.

\begin{figure}[H]
    \centering
    \subfigure[]{\includegraphics[width=.29\linewidth]{trexPlots/sys3l_80/Plots/lep_Pt_0.png}}%                             
    \subfigure[]{\includegraphics[width=.29\linewidth]{trexPlots/sys3l_80/Plots/lep_Pt_1.png}}%                      
    \subfigure[]{\includegraphics[width=.29\linewidth]{trexPlots/sys3l_80/Plots/Mll01}}\\                             
    \subfigure[]{\includegraphics[width=.29\linewidth]{trexPlots/sys3l_80/Plots/nJets.png}}%                          
    \subfigure[]{\includegraphics[width=.29\linewidth]{trexPlots/sys3l_80/Plots/nbJets.png}}%                          
    \subfigure[]{\includegraphics[width=.29\linewidth]{trexPlots/sys3l_80/Plots/MET.png}}\\                         
    \caption{Data/MC comparisons of the 3l pre-selection region. (a) and (b) show the $p_T$ of leptons 0 and 1, (c) shows the invariant mass of lepton 0 and 1, (d) shows the jet multiplicity, (e) the b-tagged jet multiplicity, and (f) the missing transverse energy.}
    \label{fig:presel3l}                                                                                          
\end{figure}

 %------------------------------------------------------------------------------------------

\subsection{Event MVA}
\label{subsec:sigBkgMVA}

Separate multi-variate analysis techniques (MVAs) are used in order to distinguish signal events from background for each analysis channel - 2lSS, 3l semi-leptonic (3lS), and 3l fully leptonic (3lF). Here events with three leptons are split into 3lS and 3lF based on the model described in \ref{sec:decay3l}. In particular, Boosted Decision Tree (BDT) algorithms are produced with XGBoost \cite{xgboost} are trained using the kinematics of signal and background events derived from Monte Carlo simulations. Events are weighted in the BDT training by the weight of each Monte Carlo event. 

Because the background composition differs for events with a high reconstructed Higgs $p_T$ compared to events with low reconstructed Higgs $p_T$, separate MVAs are produced for high and low $p_T$ regions. This is found to provide better significance than attempting to build an inclusive model, as demonstrated in appendix \ref{subsec:sigBkgApx}. A cutoff of 150 GeV is used. This gives a total of 6 background rejection MVAs - explicitly, 2lSS high $p_T$ , 2lSS low $p_T$, 3lS high $p_T$ , 3lS low $p_T$, 3lF high $p_T$, and 3lF low $p_T$.

The following features are used in both the high and low $p_T$ 2lSS BDTs:

/data_ceph/afwebb/higgs_diff/addFeatures/texFiles/tabNames_sigBkg2lSS.tex

While for each of the 3l BDTs, the features listed below are used for training:

\begin{table}[H]
  \begin{center}
  \begin{tabular}{ccc}
    \hline\hline
    M(lep, $E_T^{miss}$) & $M(l_0l_1)$ & $M(l_0l_1l_2)$ \\
    $M(l_0l_2)$ & $M(l_1l_2)$ & binHiggs $p_T$ 3lF \\
    binHiggs $p_T$ 3lS & $\Delta R(l_0)(l_1)$ & $\Delta R(l_0)(l_2)$ \\
    $\Delta R(l_1)(l_2)$ & decayScore & higgsRecoScore3lF \\
    higgsRecoScore3lS & jet  $\eta$ 0 & jet  $\eta$ 1 \\
    jet $\phi$ 0 & jet $\phi$ 1 & jet  $p_T$ 0 \\
    jet  $p_T$ 1 & Lepton  $\eta$ 0 & Lepton  $\eta$ 1 \\
    Lepton  $\eta$ 2 & Lepton $\phi$ 0 & Lepton $\phi$ 1 \\
    Lepton $\phi$ 2 & Lepton  $p_T$ 0 & Lepton  $p_T$ 1 \\
    Lepton  $p_T$ 2 & $E_T^{miss}$ & min $\Delta R(l_0)(jet)$ \\
    min $\Delta R(l_1)(jet)$ & min $\Delta R(l_2)(jet)$ & min $\Delta R(Lepton)(bjet)$ \\
    mjjMax frwdJet & nJets & nJets OR DL1r 60 \\
    nJets OR DL1r 70 & nJets OR DL1r 85 & topScore   \\
    \hline
  \end{tabular}
  \end{center}
  \caption{Input features used to distinguish signal and background events in the 3l channel.}
  \label{tab:sigBkg3lfeatures}
\end{table}


Modelling of each of these input features is verified in Appendix \ref{subsec:sigBkgApx} by comparing data and MC for 79.8 $fb^{-1}$. The BDTs are produced with a maximum tree depth of 6, using AUC as the target loss function. The BDT response distribution and ROC curve for each model is shown in Figures \ref{fig:sigBkgScore2lSS}-\ref{fig:sigBkgScore3lF}.

\begin{figure}[H]
  \centering
  \textbf{$2lSS$}\\
  \subfigure[]{\includegraphics[width=.48\linewidth]{sigBkgBDT/2lSS_highPt/xgb_score.png}}%
  \subfigure[]{\includegraphics[width=.48\linewidth]{sigBkgBDT/2lSS_highPt/xgb_roc.png}}\\
  \subfigure[]{\includegraphics[width=.48\linewidth]{sigBkgBDT/2lSS_lowPt/xgb_score.png}}% 
  \subfigure[]{\includegraphics[width=.48\linewidth]{sigBkgBDT/2lSS_lowPt/xgb_roc.png}}\\ 
  \caption{Output BDT scores of training and testing data for signal (blue) and background (orange) for $2lSS$ events with (a) high regressed Higgs \pt and (b) low regressed Higgs \pt. (b) and (d) show the ROC curve for the $2lSS$ high and low \pt models, respectively.}
  \label{fig:sigBkgScore2lSS}
\end{figure}

\begin{figure}[H]
  \centering
  \textbf{$3l$ - Semileptonic}\\
  \subfigure[]{\includegraphics[width=.48\linewidth]{sigBkgBDT/3lS_highPt/xgb_score.png}}%                                  
  \subfigure[]{\includegraphics[width=.48\linewidth]{sigBkgBDT/3lS_highPt/xgb_roc.png}}\\                                   
  \subfigure[]{\includegraphics[width=.48\linewidth]{sigBkgBDT/3lS_lowPt/xgb_score.png}}%                                   
  \subfigure[]{\includegraphics[width=.48\linewidth]{sigBkgBDT/3lS_lowPt/xgb_roc.png}}\\                                    
  \caption{Output BDT scores of training and testing data for signal (blue) and background (orange) for $3lS$ events with (a) high regressed Higgs \pt and (b) low regressed Higgs \pt. (b) and (d) show the ROC curve for the $3lS$ high and low \pt models, respectively.}
  \label{fig:sigBkgScore3lS}                                                                                                
\end{figure}

\begin{figure}[H]
  \centering
  \textbf{$3l$ - Fully Leptonic}\\
  \subfigure[]{\includegraphics[width=.48\linewidth]{sigBkgBDT/3lF_highPt/xgb_score.png}}%                                  
  \subfigure[]{\includegraphics[width=.48\linewidth]{sigBkgBDT/3lF_highPt/xgb_roc.png}}\\                                  
  \subfigure[]{\includegraphics[width=.48\linewidth]{sigBkgBDT/3lF_lowPt/xgb_score.png}}%                                 
  \subfigure[]{\includegraphics[width=.48\linewidth]{sigBkgBDT/3lF_lowPt/xgb_roc.png}}\\                                    
  \caption{Output BDT scores of training and testing data for signal (blue) and background (orange) for $3lF$ events with (a) high regressed Higgs \pt and (b) low regressed Higgs \pt. (b) and (d) show the ROC curve for the $3lF$ high and low \pt models, respectively.}
  \label{fig:sigBkgScore3lF}                                                                                                
\end{figure}

Output distributions of each MVA, comparing MC predictions to data at 79.8 $fb^{-1}$ are shown in figures \ref{fig:sigBkgScoreHigh}-\ref{fig:sigBkgScoreLow}. 

\begin{figure}
  \centering
  \textbf{High \pt Background Rejection BDTs}
  \subfigure[]{\includegraphics[width=.32\linewidth]{trexPlots/sys2l_80/Plots/sigBkg_2lSS_highPt.png}}%
  \subfigure[]{\includegraphics[width=.32\linewidth]{trexPlots/sys3l_80/Plots/sigBkg_3lS_highPt.png}}%
  \subfigure[]{\includegraphics[width=.32\linewidth]{trexPlots/sys3l_80/Plots/sigBkg_3lF_highPt.png}}\\
  \caption{Output score of the high $p_T$ BDTs in the (a) 2lSS, (b) 3lS, and (c) 3lF channels}
  \label{fig:sigBkgScoreHigh}
\end{figure}

\begin{figure}
  \centering
  \textbf{Low \pt Background Rejection BDTs}
  \subfigure[]{\includegraphics[width=.32\linewidth]{trexPlots/sys2l_80/Plots/sigBkg_2lSS_lowPt.png}}%
  \subfigure[]{\includegraphics[width=.32\linewidth]{trexPlots/sys3l_80/Plots/sigBkg_3lS_lowPt.png}}%
  \subfigure[]{\includegraphics[width=.32\linewidth]{trexPlots/sys3l_80/Plots/sigBkg_3lF_lowPt.png}}\\
  \label{fig:sigBkgScoreLow}
  \caption{Output score of the low $p_T$ BDTs in the (a) 2lSS, (b) 3lS, and (c) 3lF channels} 
\end{figure}

%------------------------------------------------------------------------------------------

\subsection{Signal Region Definitions}
\label{subsec:sigRegions}

Once pre-selection has been applied, channels are further refined based on the MVAs described above. The output of the model described in Section \ref{sec:decay3l} is used to separate the three channel into two - Semi-leptonic and Fully-leptonic - based on the predicted decay mode of the Higgs boson. This leaves three orthogonal signal regions - 2lSS, 3lS, and 3lF. 

For each event, depending on the number of leptons as well as whether the $p_T$ of the Higgs is predicted to be high ($>$ 150 GeV) or low ($<$ 150 GeV), a cut on the appropriate background rejection MVA is applied. The particular cut values, listed in Table \ref{tab:sigBkgCuts}, are determined by maximizing $S/\sqrt{B}$ in each region. 

\begin{table}[H]
  \begin{center}
  \begin{tabular}{l|c}
  \hline\hline
  Channel & BDT Score \\
  \hline
  2lSS high $p_T$ & 0.36 \\
  2lSS low $p_T$ & 0.34 \\
  3lS high $p_T$ & 0.51 \\
  3lS low $p_T$ & 0.43 \\
  3lF high $p_T$ & 0.33 \\
  3lF low $p_T$ & 0.41 \\
  \hline
  \end{tabular}
  \end{center}
  \caption{Cutoff values on background rejection MVA score applied to signal regions.}
  \label{tab:sigBkgCuts}
\end{table}

The event preselection and MVA selection listed in Table \ref{tab:sigBkgCuts} are used define the three signal regions used in the fit. These signal region definitions are summarized in Table \ref{tab:sigRegionCuts}.

\begin{table}[H]
  \begin{center}
  \begin{tabular}{ll}
  \hline\hline
  Region & Selection \\
  \hline
  2lSS & Two same charge tight leptons with $p_T$ $>$ 20 GeV \\
  & $N_{jets}\geq$ 4, $N_{b-jets}\geq$ 1 b-tagged jets \\
  & Zero $\tau_{had}$\\
  & $H^{pred}_{p_T}>$150 GeV and BDT score $>$ 0.36 \textbf{or} $H^{pred}_{p_T}<$150 GeV and BDT score $>$ 0.34 \\
  \hline
  3lS & Three light leptons with total charge $\pm 1$ \\
  & Two tight SS leptons, $p_T > 20$ GeV \\
  & One loose OS lepton, $p_T > 10$ GeV \\
  & $N_{jets}\geq$ 2, $N_{b-jets}\geq$ 1 b-tagged jets \\
  & Zero $\tau_{had}$ \\           
  & $|M(l^+l^-)-91.2\textrm{ GeV}| > 10$~\GeV{} for all OSSF lepton pairs \\
  & Decay NN Score $<$ 0.23 \\
  & $H^{pred}_{p_T}>$150 GeV and BDT score $>$ 0.51 \textbf{or} $H^{pred}_{p_T}<$150 GeV and BDT score $>$ 0.43 \\
  \hline
  3lF & Three light leptons with total charge $\pm 1$ \\
  & Two tight SS leptons, $p_T > 20$ GeV \\                                                                                  
  & One loose OS lepton, $p_T > 10$ GeV \\                                                                                   
  & $N_{jets}\geq$ 2, $N_{b-jets}\geq$ 1 b-tagged jets \\
  & Zero $\tau_{had}$ \\
  & $|M(l^+l^-)-91.2\textrm{ GeV}| > 10$~\GeV{} for all OSSF lepton pairs \\                                                 
  & Decay NN Score $>$ 0.23 \\
  & $H^{pred}_{p_T}>$150 GeV and BDT score $>$ 0.33 \textbf{or} $H^{pred}_{p_T}<$150 GeV and BDT score $>$ 0.41 \\
  \hline\hline
  \end{tabular}
  \end{center}
  \caption{Selection applied to define the three signal regions used in the fit.}
  \label{tab:sigRegionCuts}
\end{table}

