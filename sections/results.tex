Unblinded results are shown for the 80 $fb^{-1}$ data set, as well as MC only projections of results using the full Run-2, 140 $fb^{-1}$ dataset.

%-------------------------------------------
\subsection{Results - 80 $fb^{-1}$}
\label{sec:res80}
%-------------------------------------------

A maximum likelihood fit is performed simultaneously over the reconstructed Higgs $p_T$ spectrum in the three signal regions, 2lSS, 3lS, and 3lF, shown in figure \ref{fig:sigRegions80}. The $t\bar{t}H$ MC is split into high and low $p_T$, based on whether the truth $p_T$ of the Higgs is above or below 150 GeV. The parameters $\mu_{t\bar{t}H high p_T}$ and $\mu_{t\bar{t}H low p_T}$, where $\mu = \sigma_{observed}/\sigma_{SM} $, are extracted from the fit.

As the data collected from 2015-2017 has been unblinded for $t\bar{t}H-ML$ channels, representing 80 $fb^{-1}$, those events are unblinded.

\begin{figure}[H]
    \subfigure[]{\includegraphics[width=.32\linewidth]{trexPlots/stat_80/Plots/recoHiggsPt_2lSS_postFit.png}}%   
    \subfigure[]{\includegraphics[width=.32\linewidth]{trexPlots/stat_80/Plots/recoHiggsPt_3lS_postFit.png}}%    
    \subfigure[]{\includegraphics[width=.32\linewidth]{trexPlots/stat_80/Plots/recoHiggsPt_3lF_postFit.png}}\\
    \caption{Post-fit distributions of the reconstructed Higgs $p_T$ in the three signal regions, (a) 2lSS, (b) 3lS, and (c) 3lF, for 80 $fb^{-1}$ of MC}
    \label{fig:sigRegions80}
\end{figure}

A post-fit summary of the fitted regions is shown in figure \ref{fig:Summary80}.

\begin{figure}[H]
    \center
    \includegraphics[width=.9\linewidth]{trexPlots/stat_80/Plots/Summary_postFit.png}
    \caption{Post-fit summary of the yields in each signal region.}                                            
    \label{fig:Summary80}
\end{figure}

The $\mu$ values for high and low $p_T$ Higgs are shown in \ref{tab:mu80}.

\begin{table}[H]                                                                                                             
  \centering                                                                                                              
  \begin{tabular}{c}                                                                                            
     $\mu_{t\bar{t}H\ high\ p_T} = 1.0^{+0.0}_{-0.0}(stat)^{+0.0}_{-0.0}(sys)$ \\       
     $\mu_{t\bar{t}H\ low\ p_T} = 1.0^{+0.0}_{-0.0}(stat)^{+0.0}_{-0.0}(sys)$ \\
  \end{tabular}                                                                                                            
  \caption{Best fit $\mu$ values for $t\bar{t}H\ high\ p_T$ and  $t\bar{t}H\ low\ p_T$, where $\mu = \sigma_{obs}/\sigma_{pred}$}
  \label{tab:mu80}                                                                                                  
\end{table}  

\textbf{Need to add something about systematics here}

The background composition of each of the fit regions is shown in figure \ref{fig:pieChart80}.

\begin{figure}[H]
    \centering
    \includegraphics[width=0.7\linewidth]{trexPlots/stat_80/PieChart_postFit.png}
    \caption{Background composition of the fit regions.}
    \label{fig:pieChart80}
\end{figure} 

%-------------------------------------------

%-------------------------------------------                                                                                 
\subsection{Projected Results - 140 $fb^{-1}$}   
\label{sec:res140}
%------------------------------------------- 

As data collected in 2018 has not yet been unblinded for $t\bar{t}H-ML$ at the time of this note, data from that year remains blinded. Instead, an Asimov fit is performed - with the MC prediction being used both as the SM prediction as well as the data in the fit - in order to give expected results.

\begin{figure}[H]
    \subfigure[]{\includegraphics[width=.29\linewidth]{trexPlots/stat_140/Plots/recoHiggsPt_2lSS_postFit.png}}%             
    \subfigure[]{\includegraphics[width=.29\linewidth]{trexPlots/stat_140/Plots/recoHiggsPt_3lS_postFit.png}}%        
    \subfigure[]{\includegraphics[width=.29\linewidth]{trexPlots/stat_140/Plots/recoHiggsPt_3lF_postFit.png}}\\
    \caption{Blinded post-fit distributions of the reconstructed Higgs $p_T$ in the three signal regions, (a) 2lSS, (b) 3lS, and (c) 3lF, for 140 $fb^{-1}$ of data}
    \label{fig:sigRegions140}
\end{figure}

\begin{figure}[H]
    \center
    \includegraphics[width=.9\linewidth]{trexPlots/stat_140/Plots/Summary_postFit.png}
    \caption{Post-fit summary of fit.}
    \label{fig:Summary140}
\end{figure}

The $\mu$ values for high and low $p_T$ Higgs are shown in \ref{tab:mu140}.

\begin{table}[H]
  \centering
  \begin{tabular}{c}
     $\mu_{t\bar{t}H high p_T} = 1.^{+0.}_{-0.}(stat)^{+0.}_{-0.}(sys)$ \\
     $\mu_{t\bar{t}H low p_T} = 1.^{+0.}_{-0.}(stat)^{+0.}_{-0.}(sys)$ \\
  \end{tabular}
  \caption{Best fit $\mu$ values for $t\bar{t}H\ high\ p_T$ and  $t\bar{t}H\ low\ p_T$, where $\mu = \sigma_{obs}/\sigma_{pred}$} 
  \label{tab:mu140}
\end{table}

\begin{figure}[H]
    \centering                                                                                                               
    \includegraphics[width=0.7\linewidth]{trexPlots/stat_140/PieChart_postFit.png}
    \caption{Background composition of the fit regions.}
    \label{fig:pieChart140}
\end{figure}

%-------------------------------------------
