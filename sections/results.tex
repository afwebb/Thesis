A maximum likelihood fit is performed simultaneously over the reconstructed Higgs $p_T$ spectrum in the three signal regions, $2lSS$, $3lS$, and $3lF$. The signal is split into high and low $p_T$ samples, based on whether the truth $p_T$ of the Higgs is above or below 150 GeV. The parameters $\mu_{t\bar{t}H\ high\ p_T}$ and $\mu_{t\bar{t}H\ low\ p_T}$, where $\mu = \sigma_{observed}/\sigma_{SM} $, are extracted from the fit, signifying the difference between the observed value and the theory prediction. Unblinded results are shown for the 79.8 $fb^{-1}$ data set, as well as MC only projections of results using the full Run-2, 139 $fb^{-1}$ dataset.

As described in Section \ref{sec:sys}, there are 229 systematic uncertainties that are considered as NPs in the fit. These NP s are constrained by Gaussian or log-normal probability density functions. The latter are used for normalisation factors to ensure that they are always positive. The expected number of signal and background events are functions of the likelihood. The prior for each NP is added as a penalty term, decreasing the likelihood as it is shifted away from its nominal value.

%-------------------------------------------
\subsection{Results - 79.8 $fb^{-1}$}
\label{sec:res80}
%-------------------------------------------

As the data collected from 2015-2017 has been unblinded for $t\bar{t}H-ML$ channels, representing 79.8 $fb^{-1}$, those events are unblinded. The predicted Higgs $p_T$ spectrum is fit to data simultaneously in each of the three signal regions shown in Figure \ref{fig:sigRegions80}.

\begin{figure}[H]
    \centering
    \subfigure[]{\includegraphics[width=.32\linewidth]{trexPlots/sys_80/Plots/recoHiggsPt_2lSS_postFit.png}}%   
    \subfigure[]{\includegraphics[width=.32\linewidth]{trexPlots/sys_80/Plots/recoHiggsPt_3lS_postFit.png}}%    
    \subfigure[]{\includegraphics[width=.32\linewidth]{trexPlots/sys_80/Plots/recoHiggsPt_3lF_postFit.png}}\\
    \caption{Post-fit distributions of the reconstructed Higgs $p_T$ in the three signal regions, (a) $2lSS$, (b) $3lS$, and (c) $3lF$, for 79.8 $fb^{-1}$ of MC}
    \label{fig:sigRegions80}
\end{figure}

A post-fit summary of the fitted regions is shown in Figure \ref{fig:Summary80}.

\begin{figure}[H]
    \center
    \includegraphics[width=.9\linewidth]{trexPlots/sys_80/Plots/Summary_postFit.png}
    \caption{Post-fit summary of the yields in each signal region.}                                            
    \label{fig:Summary80}
\end{figure}

The the measured $\mu$ values for high and low $p_T$ Higgs production obtained from the fit are shown in \ref{tab:mu80}. A significance of 1.7$\sigma$ is observed for $t\bar{t}H\ high\ p_T$, and 2.1$\sigma$ is measured for $t\bar{t}H\ low\ p_T$.

\begin{table}[H]                                                                                                             
  \centering                                                                                                              
  \begin{tabular}{c}                                                                                            
     $\mu_{t\bar{t}H\ high\ p_T} = 2.1^{+0.62}_{-0.59}(stat)^{+0.40}_{-0.43}(sys)$ \\       
     \\
     $\mu_{t\bar{t}H\ low\ p_T} = 0.83^{+0.39}_{-0.40}(stat)^{+0.51}_{-0.53}(sys)$ \\
  \end{tabular}                                                                                                            
  \caption{Best fit $\mu$ values for $t\bar{t}H\ high\ p_T$ and  $t\bar{t}H\ low\ p_T$, where $\mu = \sigma_{obs}/\sigma_{pred}$}
  \label{tab:mu80}                                                                                                  
\end{table}  

The most prominent sources of systematic uncertainty, as measured by their impact on $\mu_{t\bar{t}H\ high\ p_T}$, are summarized in Table \ref{tab:systematics_high_80}.

\begin{table}[H]                                                                                                             
    \centering
    \begin{tabular}{l|cc}                                                                                                    
        \hline\hline                                                                                                         
        Uncertainty Source & \multicolumn{2}{c}{$\Delta \mu$ }  \\
        \hline
        Jet Energy Scale & 0.25 & 0.23 \\
        $t\bar{t}H$ cross-section (QCD scale) & -0.11 & 0.21 \\
        Luminosity & -0.13 & 0.14 \\
        Flavor Tagging & 0.14 & 0.13 \\
        $t\bar{t}W$ cross-section (QCD scale) & -0.12 & 0.11 \\
        Higgs Branching Ratio & -0.1 & 0.11 \\
        $t\bar{t}H$ cross-section (PDF) & -0.07 & 0.08 \\
        Electron ID & -0.06 & 0.06 \\
        Non-prompt Muon Normalization & -0.05 & 0.06 \\
        $t\bar{t}Z$ cross-section (QCD scale) & -0.05 & 0.05 \\
        Diboson cross-section & -0.05 & 0.05 \\
        Fake muon modelling & -0.04 & 0.04 \\
        \hline
        Total & 0.40 & 0.43 \\
        \hline\hline
    \end{tabular}
    \caption{Summary of the most significant sources of systematic uncertainty on the measurement of $t\bar{t}H\ high\ p_T$.}
    \label{tab:systematics_high_80}
\end{table}

The most significant sources of uncertainty on the measurement of $t\bar{t}H$ - low $p_T$ are shown in Table \ref{tab:systematics_low_80}.

\begin{table}[H]
    \centering
    \begin{tabular}{l|cc}
        \hline\hline
        Uncertainty Source & \multicolumn{2}{c}{$\Delta \mu$ }  \\
        \hline
        Internal Conversions & -0.26 & 0.26 \\
        Luminosity & -0.16 & 0.17 \\
        Non-prompt Muon Normalization & -0.16 & 0.16 \\
        $t\bar{t}W$ cross-section (QCD scale) & -0.17 & 0.15 \\
        Jet Energy Scale & 0.15 & 0.15 \\
        Non-prompt Electron Modelling & -0.13 & 0.14 \\
        Flavor Tagging & 0.13 & 0.13 \\
        Non-prompt Muon Modelling & -0.12 & 0.13 \\
        Non-prompt Electron Normalization & -0.11 & 0.11 \\
        $t\bar{t}Z$ cross-section (QCD scale) & -0.08 & 0.09 \\
        Diboson Cross-section & -0.07 & 0.07 \\
        \hline           
        Total & 0.51 & 0.53 \\
        \hline\hline
    \end{tabular}
    \caption{Summary of the most significant sources of systematic uncertainty on the measurement of $t\bar{t}H\ low\ p_T$.}
    \label{tab:systematics_low_80}
\end{table}

The ranking and impact of those nuisance parameters with the largest contribution to the overall uncertainty is shown in Figure \ref{fig:ranking_80}.

\begin{figure}[H]
    \centering
    \includegraphics[width=0.5\linewidth]{trexPlots/sys_80/Ranking_mu_XS_ttH_highPt.png}%
    \includegraphics[width=0.5\linewidth]{trexPlots/sys_80/Ranking_mu_XS_ttH_lowPt.png}
    \caption{Impact of systematic uncertainties on the measurement of high $p_T$ (left) and low $p_T$ (right) $t\bar{t}H$ events}
    \label{fig:ranking_80}
\end{figure}

%\begin{figure}[H]
%    \centering
%    \includegraphics[width=1.0\linewidth]{trexPlots/sys_80/CorrMatrix.png}
%    \caption{Correlations between nuisance parameters}
%    \label{fig:corr_mat_80}                                                                                                
%\end{figure}

%The background composition of each of the fit regions is shown in Figure \ref{fig:pieChart80}.

%\begin{figure}[H]
%    \centering
%    \includegraphics[width=0.7\linewidth]{trexPlots/sys_80/PieChart_postFit.png}
%    \caption{Background composition of the fit regions.}
 %   \label{fig:pieChart80}
%\end{figure} 

%-------------------------------------------

%-------------------------------------------                                                                                 
\subsection{Projected Results - 139 $fb^{-1}$}   
\label{sec:res140}
%------------------------------------------- 

As data collected in 2018 has not yet been unblinded for $t\bar{t}H-ML$ at the time of this note, data from that year remains blinded. Instead, an Asimov fit is performed - with the MC prediction being used both as the SM prediction as well as the data in the fit - in order to give expected results.

\begin{figure}[H]
    \centering
    \subfigure[]{\includegraphics[width=.29\linewidth]{trexPlots/sys_140/Plots/recoHiggsPt_2lSS_postFit.png}}%             
    \subfigure[]{\includegraphics[width=.29\linewidth]{trexPlots/sys_140/Plots/recoHiggsPt_3lS_postFit.png}}%        
    \subfigure[]{\includegraphics[width=.29\linewidth]{trexPlots/sys_140/Plots/recoHiggsPt_3lF_postFit.png}}\\
    \caption{Blinded post-fit distributions of the reconstructed Higgs $p_T$ in the three signal regions, (a) $2lSS$, (b) $3lS$, and (c) $3lF$, for 139 $fb^{-1}$ of data}
    \label{fig:sigRegions140}
\end{figure}

\begin{figure}[H]
    \center
    \includegraphics[width=.9\linewidth]{trexPlots/sys_140/Plots/Summary_postFit.png}
    \caption{Post-fit summary of fit.}
    \label{fig:Summary140}
\end{figure}

Projected uncertainties on the $\mu$ values extracted from the fit for high and low $p_T$ Higgs are shown in \ref{tab:mu140}. A significance of 2.0$\sigma$ is expected for $t\bar{t}H\ high\ p_T$, and a projected significance 2.3$\sigma$ is extracted for $t\bar{t}H\ low\ p_T$.

\begin{table}[H]
  \centering
  \begin{tabular}{c}
     $\mu_{t\bar{t}H\ high\ p_T} = 1.00^{+0.45}_{-0.43}(stat)^{+0.30}_{-0.31}(sys)$ \\
     \\
     $\mu_{t\bar{t}H\ low\ p_T} = 1.00^{+0.29}_{-0.30}(stat)^{+0.48}_{-0.50}(sys)$ \\
  \end{tabular}
  \caption{Best fit $\mu$ values for $t\bar{t}H\ high\ p_T$ and  $t\bar{t}H\ low\ p_T$, where $\mu = \sigma_{obs}/\sigma_{pred}$} 
  \label{tab:mu140}
\end{table}

The most prominent sources of systematic uncertainty, as measured by their impact on $\mu_{t\bar{t}H\ high\ p_T}$, are summarized in Table \ref{tab:systematics_high_140}.

\begin{table}[H]
    \centering
    \begin{tabular}{l|cc}                                                                                                    
        \hline\hline
        Uncertainty Source & \multicolumn{2}{c}{$\Delta \mu$ }  \\
        \hline
        Jet Energy Scale & 0.19 & 0.17 \\
        $t\bar{t}W$ Cross-section (QCD Scale) & -0.12 & 0.11 \\
        Luminosity & -0.1 & 0.11 \\
        Flavor Tagging & 0.1 & 0.1 \\
        $t\bar{t}H$ Cross-section (QCD Scale) & -0.05 & 0.1 \\
        $t\bar{t}Z$ Cross-section (QCD Scale) & -0.05 & 0.06 \\
        Non-prompt Muon Normalization & -0.05 & 0.05 \\
        Higgs Branching Ratio & -0.05 & 0.05 \\
        Diboson Cross-section & -0.04 & 0.05 \\
        Non-prompt Muon Modelling & -0.04 & 0.04 \\
        $t\bar{t}H$ Cross-section (PDF) & -0.03 & 0.04 \\
        Electron ID & -0.04 & 0.04 \\
        $t\bar{t}W$ Cross-section (PDF) & -0.03 & 0.03 \\
        \hline
        Total & 0.30 & 0.31 \\
        \hline\hline
    \end{tabular}
    \caption{Summary of the most significant sources of systematic uncertainty on the measurement of $t\bar{t}H\ high\ p_T$.}
    \label{tab:systematics_high_140}
\end{table}

The most significant sources of systematic uncertainty on $t\bar{t}H$ low $p_T$ are summarized in Table \ref{tab:systematics_low_140}.

\begin{table}[H]
    \centering
    \begin{tabular}{l|cc}
        \hline\hline
        Uncertainty Source & \multicolumn{2}{c}{$\Delta \mu$ }  \\
        \hline
        Internal Conversions & -0.18 & 0.2 \\
        Jet Energy Scale & 0.19 & 0.16 \\
        Non-prompt Muon Normalization & -0.16 & 0.17 \\
        Luminosity & -0.15 & 0.17 \\
        $t\bar{t}W$ Cross-section (QCD Scale) & -0.17 & 0.15 \\
        Non-prompt Electron Modelling & -0.13 & 0.14 \\
        Non-prompt Muon Modelling & -0.13 & 0.13 \\
        Flavor Tagging & 0.13 & 0.12 \\
        Non-prompt Electron Normalization & -0.1 & 0.11 \\
        $t\bar{t}Z$ Cross-section (QCD Scale) & -0.07 & 0.09 \\
        $t\bar{t}H$ Cross-section (QCD Scale) & -0.05 & 0.1 \\
        \hline
        Total & 0.48 & 0.50 \\
        \hline\hline
    \end{tabular}
    \caption{Summary of the most significant sources of systematic uncertainty on the measurement of $t\bar{t}H\ low\ p_T$.}
    \label{tab:systematics_low_140}
\end{table}

The ranking and impact of those nuisance parameters with the largest contribution to the overall uncertainty is shown in Figure \ref{fig:ranking_140}.
                                                                                                                             
\begin{figure}[H]
    \centering                                                                                                               
    \includegraphics[width=0.5\linewidth]{trexPlots/sys_140/Ranking_mu_XS_ttH_highPt.png}%
    \includegraphics[width=0.5\linewidth]{trexPlots/sys_140/Ranking_mu_XS_ttH_lowPt.png}
    \caption{Impact of systematic uncertainties on the measurement of high $p_T$ (left) and low $p_T$ (right) $t\bar{t}H$ events}
    \label{fig:ranking_140}
\end{figure}

%The background composition of each of the fit regions is shown in Figure \ref{fig:pieChart140}.

%\begin{figure}[H]
%    \centering                                                                                                               
%    \includegraphics[width=0.7\linewidth]{trexPlots/sys_140/PieChart_postFit.png}
%    \caption{Background composition of the fit regions.}
%    \label{fig:pieChart140}
%\end{figure}

%-------------------------------------------
