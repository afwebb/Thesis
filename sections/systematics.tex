The systematic uncertainties that are considered are summarized in Table \ref{tab:SystSummary}. These are implemented in the fit either as a normalization factors or as a shape variation or both in the signal and background estimations. The numerical impact of each of these uncertainties is outlined in section \ref{sec:results}.

\begin{table}[H]
\centering
\caption{Sources of systematic uncertainty considered in the analysis. Some of the systematic uncertainties are split into several components, as indicated by the number in the rightmost column.}
\begin{tabular}{lr}
\hline\hline
Systematic uncertainty & Components           \\
\hline
\hline
Luminosity      & 1                   \\
Pileup reweighting      & 1                   \\
\textbf {Physics Objects}       &                     \\
\ \ Electron                                    & 6                   \\
\ \ Muon        & 15                  \\
\ \ Jet energy scale and resolution     & 28                  \\
\ \ Jet vertex fraction         & 1                   \\
\ \ Jet flavor tagging          & 131                 \\
\ \ $E^{miss}_T$        & 3                   \\
\hline
Total (Experimental)        & 186                    \\
\hline
\hline
\textbf {Background Modeling}           &                     \\
\ \ Cross section                       & 24                  \\
\ \ Renormalization and factorization scales    & 10                  \\
\ \ Parton shower and hadronization model               & 2                   \\
\ \ Shower tune                         & 4                   \\
\hline
Total (Signal and background modeling)       & 40                    \\
\hline
\hline
\textbf {Background Modeling}           &                     \\
\ \ Cross section                       & 24                  \\
\ \ Renormalization and factorization scales    & 10                  \\
\ \ Parton shower and hadronization model               & 2                   \\
\ \ Shower tune                         & 4                   \\
\hline
Total (Signal and background modeling)       & 40                    \\
\hline\hline
Total (Overall)                             & 226             \\
\hline\hline
\end{tabular}
\label{tab:SystSummary}
\end{table}

The uncertainty in the combined integrated luminosity is derived from a calibration of the luminosity scale using x-y beam-separation scans performed for 13 TeV proton-proton data \cite{lumi}, \cite{LUCID2}.

The experimental uncertainties are related to the reconstruction and identification of light leptons and and b-tagging of jets, and to the reconstruction of $E^{miss}_T$. 

The sources which contribute to the uncertainty in the jet energy scale \cite{jes} are decomposed into uncorrelated components and treated as independent sources in the analysis. This method decomposes the uncertainties into 30 nuisance parameters included in the fit. A similar method is used to account for jet energy resolution (JER) uncertainties, and 8 JER uncertainty components are included as NPs in the fit.

The uncertainties in the b-tagging efficiencies measured in dedicated calibration analyses \cite{btag_cal} are also decomposed into uncorrelated components. The large number of components for b-tagging is due to the calibration of the distribution of the BDT discriminant for each of the b-tag Working Points considered in the analysis.

As mentioned in Section \ref{sec:MCsamples}, a normalization corrections and uncertainties on the estimates of non-prompt leptons backgrounds are derived using data driven techniques, described in detail in \cite{ttH_paper}. These are derived from a likelihood fit over various non-prompt enriched control regions, targeting several sources of non-prompt light leptons separately: external conversion electrons, internal conversion electrons, electrons from heavy flavor decays, and muons from heavy flavor decays. %These are used to derive overall fake factors for electrons from light source (e.g. photon conversions or light hadrons), electrons from heavy flavor decays (namely, charm or bottom hadrons), and a single fake factor for muons.

The normalization factor and uncertainty applied to each source of non-prompt leptons is summarized in Table \ref{tab:fakeNF}

\begin{table}[H]
\begin{center}
\begin{tabular}{c|c}
\hline\hline
Process &  Normalization Factor\\
\hline
$NF_e^{ExtCO}$ & 1.70 $\pm$ 0.51 \\
$NF_e^{IntCO}$ & 0.75 $\pm$ 0.26 \\
$NF_e^{HF}$ & 1.09 $\pm$ 0.32 \\
$NF_{\mu}^{HF}$ & 1.28 $\pm$ 0.17 \\
\hline
\end{tabular}
\label{tab:fakeNF}
\caption{Normalization factors - with statistical and systematic uncertainties - derived from the fit over fake control regions for each source of non-prompt leptons considered.}
\end{center}
\end{table}


In addition to those derived from the control regions, several additional uncertainties are assigned to the non-prompt lepton background. An additional 25\% uncertainty on material conversions is assigned, based on the comparison between data and MC in a region where a loose electron fails the photon conversion veto. A shape uncertainty of 15\% (6\%) is assigned to the HF non-prompt electron (muon) background based on a comparison between data and MC where the second leading electron (muon) is only required to be loose. As the contribution from light non-prompt leptons is small, about 10\% percent of the contribution from HF non-prompt leptons, it is derived from the agreement between data and simulation in a LF enriched region at low values of the non-prompt lepton BDT. The resulting uncertainty is 100\%, and is taken to be uncorrelated between internal and material conversions.

Theoretical uncertainties applied to MC predictions, including cross section, PDF, and scale uncertainties are taken from theory calculations for the predominate prompt backgrounds. Following the nominal $t\bar{t}H-ML$ analysis, a 50\% uncertainty is applied to Diboson to account for the large uncertainty in estimating VV + heavy flavor. The other ``rare'' background processes - including $tZ$, rare top processes, $ttWW$, $WtZ$, $VVV$, $tHjb$ and $WtH$ - are assigned an overall 50\% normalization uncertainty as well. The theory uncertainties applied to the MC estimates are summarized in Table \ref{tab:xsecUnc}.

\begin{table}[H]                                                                                                              {\footnotesize
\centering
../ttHDiff-PUB-Note/sections/ttH_xsecUnc.tex
\caption{Summary of theoretical uncertainties for MC predictions in the analysis.}
\label{tab:xsecUnc}}
\end{table}

Additional uncertainties to account for $t\bar{t}W$ mismodelling are also applied. These include a ``Generator'' uncertainty, based on a comparison between the nominal Sherpa 2.2.5 sample, and the formerly used aMC@NLO sample, and an ``Extra radiation'' uncertainty, which includes renormalization and factorization scale variations of the Sherpa 2.2.5 sample.
