The systematic uncertainties that are considered are summarized in Table \ref{tab:SystSummary}. These are implemented in the fit either as a normalization factors or as a shape variation or both in the signal and background estimations. The numerical impact of each of these uncertainties is outlined in section \ref{sec:results}.

\begin{table}[H]
\centering
\caption{Sources of systematic uncertainty considered in the analysis. Some of the systematic uncertainties are split into several components, as indicated by the number in the rightmost column.}
\begin{tabular}{lr}
\hline\hline
Systematic uncertainty & Components           \\
\hline
\hline
Luminosity      & 1                   \\
Pileup reweighting      & 1                   \\
\textbf {Physics Objects}       &                     \\
\ \ Electron                                    & 6                   \\
\ \ Muon        & 15                  \\
\ \ Jet energy scale and resolution     & 28                  \\
\ \ Jet vertex fraction         & 1                   \\
\ \ Jet flavor tagging          & 131                 \\
\ \ $E^{miss}_T$        & 3                   \\
\hline
Total (Experimental)        & 186                    \\
\hline
\hline
\textbf {Background Modeling}           &                     \\
\ \ Cross section                       & 24                  \\
\ \ Renormalization and factorization scales    & 10                  \\
\ \ Parton shower and hadronization model               & 2                   \\
\ \ Shower tune                         & 4                   \\
\hline
Total (Signal and background modeling)       & 40                    \\
\hline
\hline
\textbf {Background Modeling}           &                     \\
\ \ Cross section                       & 24                  \\
\ \ Renormalization and factorization scales    & 10                  \\
\ \ Parton shower and hadronization model               & 2                   \\
\ \ Shower tune                         & 4                   \\
\hline
Total (Signal and background modeling)       & 40                    \\
\hline\hline
Total (Overall)                             & 226             \\
\hline\hline
\end{tabular}
\label{tab:SystSummary}
\end{table}

The uncertainty in the combined 2015+2016 integrated luminosity is derived from a calibration of the luminosity scale using x-y beam-separation scans performed in August 2015 and May 2016 \cite{lumi}.

The experimental uncertainties are related to the reconstruction and identification of light leptons and and b-tagging of jets, and to the reconstruction of $E^{miss}_T$. The \verb!TOTAL! electron ID correlation model is used, corresponding to 1 electron ID systematic. Electron ID is found to be a subleading systematic that is unconstrained by the fit, making it an appropriate choice for this analysis.

The sources which contribute to the uncertainty in the jet energy scale \cite{jes} are decomposed into uncorrelated components and treated as independent sources in the analysis. The CategoryReduction model is used to account for JES uncertainties, which decomposes the uncertainties into 30 nuiscance parameters included in the fit. The SimpleJER model is used to account for jet energy resolution (JER) uncertainties, and 8 JER uncertainty components uncluded as NPs in the fit.

The uncertainties in the b-tagging efficiencies measured in dedicated calibration analyses \cite{btag_cal} are also decomposed into uncorrelated components. The large number of components for b-tagging is due to the calibration of the distribution of the BDT discriminant.

The systematic uncertainties associated with the signal and background processes are accounted for by varying the cross-section of each process within its uncertainty.

The full list of systematic uncertainties considered in the analysis is summarized in Tables
\ref{Tab:LeptonExperimentalSyst}, \ref{Tab:JetsExperimentalSyst} and \ref{Tab:BTagExperimentalSyst}.

\hspace{-1in}\begin{table}[H]
  \begin{center}
    {\small
    \begin{tabular}{|llcl|}
      \hline
      \multicolumn{4}{|c|}{\textbf{ Experimental Systematics on Leptons and $E_T^{miss}$} }\\
     % \hline
      Type     & Description  & Systematics Name & Application \\
     \hline
     \hline
     \multicolumn{4}{|c|}{\textbf{Trigger}}\\
     \hline
    Scale Factors    & Trigger Efficiency        & lepSFTrigTight$\_$MU(EL)$\_$SF$\_$Trigger$\_$STAT(SYST)    & Event Weight      \\
      \hline
      \multicolumn{4}{|c|}{\textbf{Muons}} \\
      \hline
      Efficiencies   & Reconstruction and        & lepSFObjTight$\_$MU$\_$SF$\_$ID$\_$STAT(SYST)              & Event Weight       \\
     & Identification    &       &        \\
      & Isolation                 &       lepSFObjTight$\_$MU$\_$SF$\_$Isol$\_$STAT(SYST)            & Event Weight       \\
         & Track To Vertex       & lepSFObjTight$\_$MU$\_$SF$\_$TTVA$\_$STAT(SYST )           & Event Weight       \\
    & Association                &                                                            &           \\
     \pt Scale   & \pt Scale & MUONS$\_$SCALE    & \pt Correction     \\
     &   &   &           \\
      Resolution     & Inner Detector            & MUONS$\_$ID                                                & \pt Correction     \\
         & Energy Resolution             &     &         \\
    & Muon Spectrometer          & MUONS$\_$MS      & \pt Correction     \\
     & Energy Resolution         &       &        \\
     &   &   &         \\
     \hline
     \multicolumn{4}{|c|}{\textbf{Electrons}}\\
     \hline
     Efficiencies    & Reconstruction            & lepSFObjTight$\_$EL$\_$SF$\_$ID                            & Event Weight              \\
     & Identification   & lepSFObjTight$\_$EL$\_$SF$\_$Reco                           & Event Weight            \\
        & Isolation                 & lepSFObjTight$\_$EL$\_$SF$\_$Isol                       & Event Weight        \\
       &   &   &          \\
     Scale Factor    & Energy  Scale             & EG$\_$SCALE$\_$ALL                                         & Energy Correction    \\
                     &   &   &          \\
     Resolution      & Energy Resolution         & EG$\_$RESOLUTION$\_$ALL                                    & Energy Correction     \\
                     &   &   &             \\
     \hline
     \multicolumn{4}{|c|}{\textbf{$E_T^{miss}$}}\\
     \hline
     Soft Tracks Terms         &             Resolution                   &      MET$\_$SoftTrk$\_$ResoPerp       &   \pt Correction  \\
                               &             Resolution                   &      MET$\_$SoftTrk$\_$ResoPara        &    \pt Correction    \\
                               &             Scale                        &      MET$\_$SoftTrk$\_$ScaleUp         &   \pt Correction     \\
                               &             Scale                        &      MET$\_$SoftTrk$\_$ScaleDown         &   \pt Correction     \\

     \hline

    \end{tabular}
   }
   \caption{\label{Tab:LeptonExperimentalSyst} Summary of experimental systematics considered for leptons and $E_T^{miss}$. Includes type, description, name of systematic as used in the fit, and mode of application. The mode of application indicates the systematic evaluation, e.g. as an  overall event re-weighting (Event Weight) or rescaling (\pt Correction).}
  \end{center}
\end{table}


\begin{table}[H]
  \begin{center}
    {\small
    \begin{tabular}{|llcc|}
      \hline
      \multicolumn{4}{|c|}{\textbf{ Experimental Systematics on Jets}} \\
      \hline
      Type     & Origin   & Systematics Name  & Application \\
      \hline
      Jet Vertex Tagger         &     & JVT      &        Event Weight          \\
        &   &   &     \\
      Energy Scale              & Calibration Method              & JET$\_$21NP$\_$           &      \pt Correction         \\       
      &   & JET$\_$EffectiveNP$\_$1-19     &    \pt Correction  \\
       &   &   &       \\
        & $\eta$ inter-calibration        & JET$\_$EtaIntercalibration$\_$Modelling    & \pt Correction          \\
     &                                 & JET$\_$EtaIntercalibration$\_$NonClosure   & \pt Correction      \\
     &                                 & JET$\_$EtaIntercalibration$\_$TotalStat    & \pt Correction      \\
    &   &   &        \\
     & High \pt jets                   & JET$\_$SingleParticle$\_$HighPt         &     \pt Correction             \\
        &   &   &           \\
        & Pile-Up                         & JET$\_$Pileup$\_$OffsetNPV            &     \pt Correction             \\
        &       & JET$\_$Pileup$\_$OffsetMu             &     \pt Correction               \\
        &        & JET$\_$Pileup$\_$PtTerm       &     \pt Correction         \\
        &                                         & JET$\_$Pileup$\_$RhoTopology      &     \pt Correction             \\
        &   &   &            \\
          & Non Closure                     & JET$\_$PunchThrough$\_$MC15    & \pt Correction    \\
        &   &   &       \\
         & Flavour                         & JET$\_$Flavor$\_$Response          &   \pt Correction            \\
     &         & JET$\_$BJES$\_$Response          &   \pt Correction           \\
           &                                 & JET$\_$Flavor$\_$Composition        &    \pt Correction             \\
                &   &   &          \\
      Resolution                &                                 & JET$\_$JER$\_$SINGLE$\_$NP          &  Event Weight       \\
                                &   &   &          \\

    \hline

     \end{tabular}
    }
    \caption{\label{Tab:JetsExperimentalSyst} Jet systematics take into account effects of jets calibration method, $\eta$ inter-calibration, high \pt jets, pile-up, and flavor response. They are all diagonalised into effective parameters.}
 \end{center}
\end{table}

\begin{table}[H]
  \begin{center}
    {\small
    \begin{tabular}{|llc|}
      \hline
     \multicolumn{3}{|c|}{\textbf{Experimental Systematics on b-tagging}} \\
      \hline
      Type     & Origin   & Systematic Name \\
     \hline
     &   &                \\
      Scale Factors & DL1r b-tagger efficiency & DL1r$\_$Continuous$\_$EventWeight$\_$B0-29 \\
      &    on b originated jets in bins of $\eta$  &   \\
      &   &                \\
      &    DL1r b-tagger efficiency & DL1r$\_$Continuous$\_$EventWeight$\_$C0-19  \\
      &    on c originated jets in bins of $\eta$    &     \\
      &   &   \\
      &    DL1r b-tagger efficiency & DL1r$\_$Continuous$\_$EventWeight$\_$Light0-79           \\
      &    on light flavoured originated jets         &   \\
     &     in bins of $\eta$ and \pt      &    \\
         &   &             \\
     &    DL1r b-tagger                        & DL1r$\_$Continuous$\_$EventWeight$\_$extrapolation  \\
     &    extrapolation efficiency    &         DL1r$\_$Continuous$\_$EventWeight$\_$extrapolation$\_$from$\_$charm             \\
     \hline
    \end{tabular}
    }
    \caption{\label{Tab:BTagExperimentalSyst} Summary of experimental systematics to be included for $b$-tagging of jets in the analysis, using the continuous DL1r tagging algorithm. All of the b-tagging related systematics are applied as event weights. From left: type, description, and the name of systematic used in the fit.}
  \end{center}
\end{table}

As mentioned in Section \ref{sec:MCsamples}, a normalization corrections and uncertainties on the estimates of non-prompt leptons backgrounds are derived using data driven techniques, decribed in detail in \cite{ttH_paper}. These are derived from a likelihood fit over various non-prompt enriched control regions, targeting several sources of non-prompt light leptons separately: external conversion electrons, internal conversion electrons, electrons from heavy flavor decays, and muons from heavy flavor decays. %These are used to derive overall fake factors for electrons from light source (e.g. photon conversions or light hadrons), electrons from heavy flavor decays (namely, charm or bottom hadrons), and a single fake factor for muons.

The normalization factor and uncertainty applied to each source of non-prompt leptons is summarized in Table \ref{tab:fakeNF}

\begin{table}[H]
\begin{center}
\begin{tabular}{c|c}
\hline\hline
Processs &  Normalization Factor\\
\hline
$NF_e^{ExtCO}$ & 1.70 $\pm$ 0.51 \\
$NF_e^{IntCO}$ & 0.75 $\pm$ 0.26 \\
$NF_e^{HF}$ & 1.09 $\pm$ 0.32 \\
$NF_{\mu}^{HF}$ & 1.28 $\pm$ 0.17 \\
\hline
\end{tabular}
\label{tab:fakeNF}
\caption{Normalization factors - with statistical and systematic uncertainties - derived from the fit over fake control regions for each source of non-prompt leptons considered.}
\end{center}
\end{table}


In addition to those derived from the control regions, several additional uncertainties are assigned to the non-prompt lepton background. An additional 25\% uncertainty on material conversions is assigned, based on the comparison between data and MC in a region where a loose electron fails the photon conversion veto. A shape uncertainty of 15\% (6\%) is assigned to the HF non-prompt electron (muon) background based on a comparison between data and MC where the second leading electron (muon) is only required to be loose. As the contribution from light non-prompt leptons is small, about 10\% percent of the contribution from HF non-prompt leptons, it is derived from the agreement between data and simulation in a LF enriched region at low values of the non-prompt lepton BDT. The resulting uncertainty is 100\%, and is taken to be uncorrelated between internal and material conversions.

Theoretical uncertainties applied to MC predictions, including cross section, PDF, and scale uncertainties are taken from theory calculations for the predominate prompt backgrounds. Following the nominal $t\bar{t}H-ML$ analysis, a 50\% uncertainty is applied to Diboson to account for the large uncertainty in estimating VV + heavy flavor. The other ``rare'' background processes - including $tZ$, rare top processes, $ttWW$, $WtZ$, $VVV$, $tHjb$ and $WtH$ - are assigned an overall 50\% normalization uncertainty as well. The theory uncertainties applied to the MC estimates are summarized in Table \ref{tab:xsecUnc}.

\begin{table}[H]                                                                                                              {\footnotesize
\centering
../ttHDiff-PUB-Note/sections/ttH_xsecUnc.tex
\caption{Summary of theoretical uncertainties for MC predictions in the analysis.}
\label{tab:xsecUnc}}
\end{table}

Additional uncertainties to account for $t\bar{t}W$ mismodelling are also applied. These include a ``Generator'' uncertainty, based on a comparison between the nominal Sherpa 2.2.5 sample, and the formerly used aMC@NLO sample, and an ``Extra radiation'' uncertainty, which includes renormalisation and factorisation scale variations of the Sherpa 2.2.5 sample.
