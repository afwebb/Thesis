Particle physics is an attempt to describe the fundamental building blocks of the universe and their interactions. The Standard Model (SM) - our best current theory of fundamental particle physics - does a remarkable job of that. All known fundamental particles and (almost) all of the forces underlying their interactions can be explained by the SM, and the predictions from this theory agree with experiment to an incredibly precise degree. This is especially true since the Higgs boson, the last piece of the SM predicted decades before, was finally discovered at the Large Hadron Collider (LHC) in 2012 \cite{HIGG-2012-27}. 

Despite the success of the SM, there remains significant work to be done. For one, the SM is incomplete: it fails to provide a description of gravity, to give an explanation for the observation of Dark Matter, or to provide a mechanism for neutrinos to gain mass. Further, a Higgs boson with a mass of around 125 GeV, as observed at the LHC, gives rise to what is known a hierarchy problem - such a low mass Higgs requires a seemingly unnatural level of ``fine tuning'' that is unexplained by the SM.

A promising avenue for addressing these problems is to study the properties of the Higgs boson and the way it interacts with other particles, in part simply because these interactions have not been measured before. Its interactions with the top quark are a particularly promising place to look. Because the Higgs field is responsible for allowing particles to acquire mass, the strength of a particle's interaction with the Higgs boson is proportional to its mass. As the most massive of the fundamental particles, the top quark has the strongest coupling to the Higgs boson, making this interaction particularly interesting to study.

These interactions can be measured by directly by studying the production of a Higgs boson in association with a pair of top quarks ($t\bar{t}H$). While studies have been done measuring the overall rate of $t\bar{t}H$ production, there are several theories of physics Beyond the Standard Model (BSM) that would affect the kinematics of $t\bar{t}H$ production without altering its overall rate. This dissertation demonstrates the feasibility of performing differential measurements of the kinematics of the Higgs boson in $t\bar{t}H$ events.

%An Effective Field Theory model can be used to model the low energy effects of high energy physics.

The proton-proton collision data collected by the ATLAS detector at the LHC from 2015-2018 provides the opportunity to make this measurement for the first time. The unprecedented energy achieved by the LHC during this period - compared to past accelerators and previous runs of the LHC itself - greatly increase the rate at which $t\bar{t}H$ events are produced, and the large amount of data collected provides the necessary statistics for a differential measurement to be performed. An exploratory study into differential measurements of the Higgs transverse momentum is presented using $t\bar{t}H$ events with multiple leptons in the final state. In addition to an explanation of the techniques used, preliminary results are shown for 79.8 $fb^{-1}$ of data from proton-proton collisions at an energy $\sqrt{s} = 13$ TeV collected by the ATLAS detector from 2015-2017. Projected results are shown for the full 139 $fb^{-1}$ of data collected from 2015-2018. 

Further, a measurement of WZ produced in association with a heavy flavor jet (including both b-jets and charm jets) is performed. This process mimics the final state of $t\bar{t}H$ multilpeton events, making it an irreducible background for that analysis and many others. However, this process is poorly understood, and difficult to simulate accurately, introducing large systematic uncertainties for analyses that include it as a background. A measurement of WZ + heavy flavor in the fully leptonic decay mode is performed in an attempt to reduce this uncertainty.

This dissertation begins with a brief explanation of the SM, its limitations, and the theoretical motivation behind this workin Part \ref{part:theory}. This is followed by a description of the LHC and the ATLAS detector in Part \ref{part:lhcAtlas}. Part \ref{part:wz} details a measurement of WZ + heavy flavor. Studies of differential measurements of $t\bar{t}H$ are then described in Part \ref{part:analysis}, and preliminary results are presented. Finally, the results of these studies are summarized in the conclusion, Part \ref{part:conclusion}.

