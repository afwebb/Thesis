Particle physics is an attempt to describe the fundamental building blocks of the universe and their interactions, and the Standard Model (SM) - our best current theory of fundamental particle physics - does a remarkable job of doing exactly that. All known fundamental particles and (almost) all of the forces underlying their interactions can be explained by the SM, and the predictions from this theory agree with experiment to an incredibly precise degree. This is especially true since the Higgs Boson, last piece of the SM predicted decades before, was finally discovered at the Large Hadron Collider (LHC) in 2012. 

Despite the success of the SM, there remains significant work to be done. For one, the SM is incomplete: it fails to explain gravity at the quantum scale, to give any explanation for the observation fo Dark Matter, or to provide a mechanism for neutrinos to gain mass. A Higgs Boson with a mass of around 125 GeV also gives rise to what is known a hierarchy problem - such a low mass Higgs requires a seemingly unnatural level of ``fine tuning''. 

A promising avenue for addressing these problems is to study the properties of the Higgs Boson and the way it interacts with other particles, in part because these interactions have not been accurately measured before. This leaves the Higgs sector of the SM a fertile ground to search for new physics. Its interactions with the Top Quark are a particularly promising place to look: The Higgs Boson is responsible for giving particles their mass, and the strength of a particle's interaction with the Higgs is proportional to its mass. As the most massive of the fundamental particles, the Top Quark has the strongest coupling to the Higgs Boson. This means any new physics in the Higgs sector is likely to present itself most prominantly in its interaction with the Top Quark.

These interactions can be measured by directly by studying the production of a Higgs Boson in association with a pair of Top Quarks ($t\bar{t}H$). While studies have been done measuring the overall rate $t\bar{t}H$ of production, there are several theories of physics Beyond the Standard Model (BSM) that would affect the kinematics of $t\bar{t}H$ production without affecting its overall rate. This dissertation attempts to make a differential measurement on the kinematics of $t\bar{t}H$ production in order to search for these BSM effects.

The proton-proton collision data collected by the ATLAS detector at the LHC from 2015-2018 provides the oppurtunity to made such a measurement for the first time. The unprecedented energy acheived by the LHC during this period greatly increase the rate at which $t\bar{t}H$ events are produced, and the large amount of data collected allow a differential measurement to be performed.

This dissertation begins with a brief explanation of the SM, its limitations, and the theoretical motivation behind this work. This is followed by a description of the LHC and the ATLAS detector. The analysis strategy is then described, and the results are presented. Finally, the results of the study are summarized in the conclusion.