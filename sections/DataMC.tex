
For both data and Monte Carlo (MC) simulations, samples were prepared in the \verb|xAOD| format, which was used to produced a \verb|xAOD| based on the \verb|HIGG8D1| derivation framework. This framework was designed for the main $t\bar{t}H$ multi-lepton analysis. Because this analysis targets events with multiple light leptons, as well as tau hadrons, this framework skims the dataset of any events that do not meet at least one of the following requirements:

\begin{itemize}
    \item at least two light leptons within a range $|\eta|$<2.6, with leading lepton $p_{T}$ > 15 GeV and subleading lepton $p_{T}$ > 5 GeV
    \item at least one light lepton with $p_{T}$ > 15 GeV within a range $|\eta|$<2.6, and at least two hadronic taus with $p_{T}$ > 15 GeV.
\end{itemize}

Samples were then generated from these \verb|HIGG8D1| derivations using AnalysisBase version 21.2.127.

\subsection{Data Samples}

The study uses proton-proton collision data collected by the ATLAS detector from 2015 through 2018, which represents an integrated luminosity of 139 $fb^{-1}$ and an energy of $\sqrt{s} = 13$ TeV. All data used in this analysis was included in one the following Good Run Lists:

\begin{itemize}
    \item data15\_13TeV.periodAllYear\_DetStatus-v79-repro20-02\_DQDefects-00-02-02\\\_PHYS\_StandardGRL\_All\_Good\_25ns.xml
    \item data16\_13TeV.periodAllYear\_DetStatus-v88-pro20-21\_DQDefects-00-02-04\\\_PHYS\_StandardGRL\_All\_Good\_25ns.xml 
    \item data17\_13TeV.periodAllYear\_DetStatus-v97-pro21-13\_Unknown\_PHYS\_StandardGRL\\\_All\_Good\_25ns\_Triggerno17e33prim.xml
    \item data18\_13TeV.periodAllYear\_DetStatus-v102-pro22-04\_Unknown\_PHYS\_StandardGRL\\\_All\_Good\_25ns\_Triggerno17e33prim.xml
\end{itemize}

\subsection{Monte Carlo Samples}

Several Monte Carlo (MC) generators were used to simulate both signal and background processes. For all of these, the effects of the ATLAS detector are simulated in Geant4. The specific event generator used for each of these MC samples is listed in table \ref{tbl:evgen}.

\begin{table}[H]
\begin{center}                                                                                                                \caption{\label{tbl:evgen} The configurations used for event generation of signal and background processes, including the event generator, matrix element (ME) order, parton shower algorithm, and parton distribution functin (PDF). }
 \resizebox{\textwidth}{!}{ 
\begin{tabular}{llllll}
\hline\hline                                                                                                               
Process & Event generator & ME order & Parton Shower & PDF   \\
\hline
$t\bar{t}H$ & \textsc{MG5\_aMC} & NLO & \textsc{Pythia} 8\ & NNPDF 3.0 NLO \cite{Ball:2014uwa} \\                          
& (\textsc{MG5\_aMC}) & (NLO) & (\textsc{Herwig++}) & (CT10 \cite{ct10})  \\
$\ttbar W$ & \textsc{MG5\_aMC} & NLO & \textsc{Pythia} 8 & NNPDF 3.0 NLO \\
& (\textsc{Sherpa} 2.1.1) & (LO multileg) & (\textsc{Sherpa}) & (NNPDF 3.0 NLO)  \\
$\ttbar (Z/\gamma^* \to ll)$ & \textsc{MG5\_aMC} & NLO & \textsc{Pythia} 8 & NNPDF 3.0 NLO  \\
$VV$ & \textsc{Sherpa} 2.2.2 & MEPS NLO & \textsc{Sherpa} & CT10 \\
$\ttbar$ & \textsc{Powheg-BOX v2} \cite{powhegtt} & NLO & \textsc{Pythia} 8 & NNPDF 3.0 NLO  \\
$\ttbar\gamma$ & \textsc{MG5\_aMC} & LO & \textsc{Pythia} 8 & NNPDF 2.3 LO \\
$t Z$ & \textsc{MG5\_aMC} & LO & \textsc{Pythia} 6  & CTEQ6L1  \\
$tHqb$ & \textsc{MG5\_aMC} & LO & \textsc{Pythia} 8 & CT10  \\
$tHW$ & \textsc{MG5\_aMC} & NLO & \textsc{Herwig++}  & CT10  \\
& (\textsc{Sherpa} 2.1.1) & (LO multileg) & (\textsc{Sherpa}) & (NNPDF 3.0 NLO)  \\                                        
$t W Z$ & \textsc{MG5\_aMC} & NLO & \textsc{Pythia} 8 & NNPDF 2.3 LO   \\
$t\bar t t$, $t\bar t t\bar t$ & \textsc{MG5\_aMC} & LO & \textsc{Pythia} 8 & NNPDF 2.3 LO  \\                             
$t\bar t W^+ W^-$ & \textsc{MG5\_aMC} & LO & \textsc{Pythia} 8 & NNPDF 2.3 LO\\
$s$-, $t$-channel, & \textsc{Powheg-BOX v1} \cite{powhegstp}& NLO & \textsc{Pythia} 6 & CT10 \\
$Wt$ single top & & & &  \\
$qqVV$, $VVV$ & &   \\
$Z \to l^+l^-$ & \textsc{Sherpa} 2.2.1 & MEPS NLO  & \textsc{Sherpa} & NNPDF 3.0 NLO \\
\hline\hline
\end{tabular}
}
\end{center}
\end{table}

While the main $t\bar{t}H$ analysis uses a more sophisticated data-driven approach to estimating the contribution of events with non-prompt leptons (or ''fakes''), at the time of this note this strategy has not been completely developed for the full Run-2 dataset. Therefore, the non-prompt contribution is estimated with MC, while applying conservative systematic uncertainties to these processes, as described in Section \ref{sec:sys}.

The specific DSIDs used in the analysis are listed below:

\begin{table}[H]
    \centering
    \begin{tabular}{l|l}
        \hline\hline
        Sample & DSID \\
        \hline\hline
        $t\bar{t}H$ & 345873-5, 346343-5 \\ 
        $VV$ & 364250-364254, 364255, 363355-60, 364890 \\
        $t\bar{t}W$ & 410155 \\
        $t\bar{t}Z$ & 410156, 410157, 410218-20 \\
        low mass $t\bar{t}Z$ & 410276-8 \\
        Rare Top & 410397, 410398, 410399 \\
        single Top & 410658-9, 410644-5 \\
        three Top & 304014 \\
        four Top & 410080 \\
        $t\bar{t}WW$ & 410081 \\
        Z + jets & 364100-41 \\
        low mass Z + jets & 364198-215 \\
        W + jets & 364156-97 \\
        $V\gamma$ & 364500-35 \\
        $tZ$  & 410560 \\
        $tW$  & 410013-4 \\
        $WtZ$ & 410408 \\
        $VVV$ & 364242-9 \\
        $VH$ & 342284-5 \\
        $WtH$ & 341998 \\
        $t\bar{t}\gamma$ & 410389 \\
        $t\bar{t}$ & 410470 \\
        \hline\hline
    \end{tabular}
    \caption{List of Monte Carlo samples by data set ID used in the analysis.}
    \label{tbl:dsids}
\end{table}
