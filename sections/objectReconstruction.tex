
All analysis channels considered in this note share a common object selection for leptons and jets, as well as a shared trigger selection. 

\subsection{Trigger Requirements}

Events are required to be selected by dilepton triggers, as summarized in table \ref{tbl:trigger}.

\begin{table}[h!]
 \begin{center}
   \begin{tabular}{cc}
     \toprule
                  & Dilepton triggers (2015) \\
     \midrule
      $\mu\mu$ (asymm.)          & \verb!HLT_mu18_mu8noL1! \\
      $ee$ (symm.)               & \verb!HLT_2e12_lhloose_L12EM10VH! \\
      $e\mu,\mu e$ ($\sim$symm.) & \verb!HLT_e17_lhloose_mu14! \\
     \bottomrule
                       & Dilepton triggers (2016) \\
     \midrule
      $\mu\mu$ (asymm.)                   & \verb!HLT_mu22_mu8noL1! \\
      $ee$ (symm.)                        & \verb!HLT_2e17_lhvloose_nod0! \\
      $e\mu,\mu e$ ($\sim$symm.)          & \verb!HLT_e17_lhloose_nod0_mu14! \\
     \bottomrule

                  & Dilepton triggers (2017) \\
     \midrule
      $\mu\mu$ (asymm.)                   & \verb!HLT_mu22_mu8noL1! \\
      $ee$ (symm.)                        & \verb!HLT_2e24_lhvloose_nod0! \\
      $e\mu,\mu e$ ($\sim$symm.)          & \verb!HLT_e17_lhloose_nod0_mu14! \\
     \bottomrule
                  & Dilepton triggers (2018) \\
     \midrule
      $\mu\mu$ (asymm.)                   & \verb!HLT_mu22_mu8noL1! \\
      $ee$ (symm.)                        & \verb!HLT_2e24_lhvloose_nod0! \\
      $e\mu,\mu e$ ($\sim$symm.)          & \verb!HLT_e17_lhloose_nod0_mu14! \\
      \bottomrule
   \end{tabular}
   \caption{\label{tbl:trigger} List of lowest $p_{T}$-threshold, un-prescaled dilepton triggers used for 2015-2018 data taking.}
 \end{center}
\end{table}

\subsection{Light Leptons}
\label{subsec:lepSelection}

Electron candidates are reconstructed from energy clusters in the electromagnetic calorimeter that are associated with charged particle tracks reconstructed in the inner detector \cite{ATLAS-CONF-2016-024}.  Electron candidates are required to have $\pt > 10$ GeV and $|\eta_\textrm{cluster}| < 2.47$. Candidates in the transition region between different electromagnetic calorimeter components, $1.37 < |\eta_\textrm{cluster}| < 1.52$, are rejected. A multivariate likelihood discriminant combining shower shape and track information is used to distinguish prompt electrons from nonprompt leptons, such as those originating from hadronic showers. 

To further reduce the non-prompt contribution, the track of each electron is required to originate from the primary vertex; requirements are imposed on the transverse impact parameter significance ($|d_0|/\sigma_{d_0}$) and the longitudinal impact parameter ($|\Delta z_0 \sin \theta_\ell|$), as shown in table \ref{tbl:tightleps}.

Muon candidates are reconstructed by combining inner detector tracks with track segments or full tracks in the muon spectrometer \cite{PERF-2014-05}. Muon candidates are required to have $\pt > 10$~GeV and $|\eta| < 2.5$. All leptons are required to be isolated, and pass a non-prompt BDT selection described in detail in \cite{ttH_paper}.

\subsection{Jets}
\label{subsec:jetSelection}

%UPDATE TO PFLOW
Jets are reconstructed from calibrated topological clusters built from energy deposits in the calorimeters \cite{ATL-PHYS-PUB-2015-015}, using the anti-$k_t$ algorithm with a radius parameter $R=0.4$.  Jets with energy contributions likely arising from noise or detector effects are removed from consideration \cite{ATLAS-CONF-2015-029}, and only jets satisfying $\pt > 25$~GeV and $|\eta| < 2.5$ are used in this analysis.  For jets with $\pt < 60$~GeV and $|\eta| < 2.4$, a jet-track association algorithm is used to confirm that the jet originates from the selected primary vertex, in order to reject jets arising from pileup collisions \cite{PERF-2014-03}. 

\subsection{Missing Transverse Energy}

Because all $t\bar{t}H-ML$ channels considered include multiple neutrinos, missing transverse energy ($E_T^{miss}$) is present in each event. The missing transverse momentum vector is defined as the inverse of the sum of the transverse momenta of all reconstructed physics objects as well as remaining unclustered energy, the latter of which is estimated from low-\pt tracks associated with the primary vertex but not assigned to a hard object \cite{ATL-PHYS-PUB-2015-027}.
