ATLAS is a general purpose detector designed to maximize the detection efficiency of nearly all physics objects, including leptons, jets, and photons, while covering nearly the entire solid angle around the collision point. Just surrounding the interaction point is the Inner Detector, designed to track the path of charged particles moving through the detector. An inner solenoid surrounding the Innder Detector is used to produces a magnetic field of 2 T.

The Inner Detector consists of three components - the Pixel Detector, the Semi-Conductor Tracker (SCT), and the Transition Radiation Tracker (TRT). The Pixel Detector is the innermost of these, beginning just 33.25 mm away from the beam line. It consists of three silicon layers along the barrel, as well as three endcap layers, covering a range of $|\eta|$ < 2.5. The Semiconductor Tracker (SCT) is similar to the Pixel detector, but uses long strips rather than small pixel to cover a larger spatial area.

Situated outside the Innder Detector are two concentric calorimeters, covering a range of $|\eta|$<4.9. The inner calorimeter uses liquid argon (LAr) to measure energy of particles that interact electromagnetically within the region $|\eta|$ < 3.2, and consists of around 180,000 readout channels.  The outer calorimeter, or hadronic calorimeter, is composed of steel plates, with scintillating tiles as the active material. It covers a range of $|\eta|$ < 1.7, and the signals from the hadronic calorimeter are read out by photomultiplier tubes (PMTs). The remain pseudorapidity range is covered by forward calorimeter modules.

The outermost layer of the detector, the muon spectrometer, consists of tracking and triggering system. It extends from the outside of the calormeter system, about a 4.25 m radius from the beam line, to a radius of 11 m. Two large toroidal magnets within the muon system generate a large magnetic ranging between 2 T and 8 T. 1200 tracking chambers are placed in the muon system in order to precisely measure the tracks of muons within $|\eta|$ < 2.7 with high spatial resolution.

A two-level trigger system is used to select out events to be recorded. The level-1 trigger uses hardware information from the calorimeters and muon spectrometer to select events that contain candidates for particles commonly used in analysis, and reduces the rate of events from 40 MHz to around 100 kHz. Events that pass the level-1 trigger move to the High-Level Trigger (HLT). The HLT takes place outside of the detector in software, and looks for properties such as a large amount of missing transverse energy, well defined leptons, and multiple high energy jets. Events that pass the HLT are stored and used for analysis. Because the specifics of the HLT are determined by software rather than hardware, the thresholds can be changed throughout the run of the detector in response to run conditions such as changes to pilup and luminosity. After the HLT is applied, the event rate is reduced to around 1 kHz, which are recorded for analysis.
