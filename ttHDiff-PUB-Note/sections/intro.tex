Since the discovery of a Higgs boson compatable with the Standard Model (SM) in 2012 \cite{}, its interactions with other particles have been studied using proton-proton collision data produced by the Large Hadron Collider (LHC). The strongest of these interactions is the coupling of the Higgs to the top quark, making the Yukawa coupling between these two particles of particular interest for study.

These interactions can be measured directly by studying the production of a Higgs Boson in association with a pair of Top Quarks ($t\bar{t}H$) \cite{}. While this process has been observed by both the ATLAS \cite{} and CMS \cite{} collaborations, these analyses have focused on measuring the overall rate of $t\bar{t}H$ production. There are several theories of physics Beyond the Standard Model (BSM), however, that would affect the kinematics of $t\bar{t}H$ production without altering its overall rate \cite{}.  

An Effective Field Theory approach can be used to model the low energy effects of new, high energy physics, by paramaterizing BSM effects as dimension-six operators. The addition of these operators can be shown to modify the transverse momentum ($p_T$) spectrum of the Higgs Boson \cite{}. Therefore, reconstructing the momentum spectrum of the Higgs provides a means to observe new physics in the Higgs sector.  

This note reports on the feasability of measuring the impact of dimension-six operators in $t\bar{t}H$ events with multiple leptons in the final state, using Monte Carlo (MC) simulations scaled to 139 $fb^{-1}$ at an energy $\sqrt{s} = 13$ TeV. Events are separated into channels based on the number of light leptons (electrons and muons) in the final state - either two same-sign leptons ($2lSS$), or three leptons ($3l$). A deep neural network is used to identify which objects originate from the decay of the Higgs, and reconstruct the momentum of the Higgs Boson in each event. This reconstructed momentum spectrum is used to place limits on BSM effects, and on the parameters of dimension-six operators.

This note is organized as follows: The dataset and Monte Carlo (MC) simulations used in the analysis is outlined in section \ref{sec:dataMC}. Section \ref{sec:objReco} describes the identification and reconstruction of the various physics objects. The models used to reconstruct the momentum spectrum of the Higgs is discussed in section \ref{sec:mva}. The selection and categorisation of events comprises section \ref{sec:signal_region}, and the theoritical and experimental systematic uncertainties considered are described in section \ref{sec:sys}. Finally, the results of the study are summarized in section \ref{sec:results}.
